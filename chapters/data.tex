\chapter{Data structure, features and hidden states}

reference json grinder and its usage to process data

reference some source that is talking about data in cyber security, resp. on theory part

- What kind of data we have - mention all parts of cuckoo log, mention even virus total reports
- (theory for basic statistcs, but maybe not)
- GOALS
  - data processing, pruning and so on
  - Analyze captured data. Report basic statistics and choose appropriate features and hidden states for further modelling
- dataset structure
  - Bias in practical data like this - security data, what are the influences
  - Balanced dataset - in term of accuracy metric performance
  - What each part means (including signatures)
- basic statistics - histogram...
- On Hmill usage - features, hidden states (Section for each, description)
- Signatures - hidden states
- Features  desription
- Maybe try to extract more candidates (But I would have to perform several more experiments, but data pruning should not be problem)
  - I tried to choose enhanced and summary part but it was too much and enhanced part contains bias like timestamp and so on, so I decided to choose only summary part and later only segments from it, using whole summary lead us to really slow convergence...I think
