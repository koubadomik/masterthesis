\chapter{Model explaining} \label{chap:expth}
In following chapter we describe model explainability in general - definition, motivation, prior arts. Finally, we give details of \emph{hmill} explainer which we use in our experiments.

\section{Motivation}
In case of complex model structures like artificial neural nets we often want to explain its predictions. For example in case of malware we may be sure that the information used by the model is quite simple and straight-forward compared to the whole sample or at least that is in our expectation. That is why we often try to explain the models results to see if the explanation (e.g. \ subset of original sample which gives the model majority of information based on which it classifies) is hand in hand with our intuition. It could also be kind of hand brake to see that the model is not generalizing based on something relevant. For example in base prior article for us authors identified that their model is classifying mainly according to timestamp field in original sample, which was different for malware and for cleanware \cite{Pevny2020}. This was obviously mistake because this detail is not the difference between malware and cleanware.

Going from popular quote by George E.P. Box: \say{All models are wrong but some of them are useful} We should ask: \emph{What kind of model is useful?}
One of the greatest advice in machine learning and statistical data analysis is that less is more. If we are modeling some unknown process based on given data we should start from the simplest model and incrementaly proceed to more complex. 
This fact is derived from \emph{Bias x Variance tradeoff} \todo{Maybe some citation, https://blog.insightdatascience.com/always-start-with-a-stupid-model-no-exceptions-3a22314b9aaa, describe better} and we can see practical consequences during model extrapolation but we can see even interpretability and explainability. That is why our next question should be: \emph{Why should we trust the model?} \cite{Ribeiro2016}.

In May 2018 General Data Protection Regulation (GDPR) became law. It has innovative clauses on automated decision-making and, to some extent, right of explanation. All individuals might enforce to obtain “meaningful explanations of the logic involved” when automated decision making takes place \cite{Guidotti2018}. That is a significant scientific challenge in the field where we face such a great boom regarding the statistical model's performance and disproportionately weak understanding of its behaviour. A lot of articles like \cite{Zhang2016} leads us to conclusion that neural net generalization is against our intuition or expectations.

This field is a little bit like a border between artificial inteligence, philosophy and other branches. We do not aspire to cover everything, the goal is to summarize terminology and basic challenges. This part of the thesis is more experimental, just to see what results we have and how we can interpret them.


Techniques of intepretation and explaining are growing in popularity as a tool for further statistical model analysis. It might lead us to better model understanding or it can shed some more light on examined domain (extract new knowledge) \cite{Montavon2018}.

\section{Definition}

Based on \cite{Montavon2018} we can define two basic terms.
\begin{definition}
An \emph{iterpretation} is the mapping $$\mathcal{Y}\rightarrow\mathcal{D}$$. $$\mathcal{Y}$$ was defined in \todo{ref chapter Classification}) as \emph{state space}, in general it can be any kind of abstract concept (real-valued vectors, sequences\dots).  $$\mathcal{D}$$ denotes domain which is human-readable and understandable (image, heatmap, sequence of words\dots). Typically $$D$$ is the same as feature space \todo{define feature space} $$\mathcal{X}$$ or its transformation.
\end{definition}

\begin{definition}
    An \emph{explanation} $$e \in \mathcal{d}$$ is a subset $$e$$ of a interpretable domain $$\mathcal{D}$$ that contributed for the predicted state $$y \in \mathcal{Y}$$ (or contributed significantly more than other members and so it is cause of the result).
\end{definition}

This definition is quite vague because human-readability and understadability is not something to measure or observe precisely. We might assume that the original \emph{state space} is not straight-forward for human understanding which is usually true e.g. \ neural net's predicted class (just 1 or 0). In case of a neural net we can see only its parameters i.e. \emph{weights} and \emph{biases}, input features and than prediction. But the question is \emph{Why?}. Fortunately, final layer (or funciton) in neural network is \emph{softmax}, so we can try to interpret class probabilites instead just 1 or 0 outputs. We might want to see the subset of input features which is responsible for the majority of accumulated class probability. This subset should be intepretable by human as mentioned in definition because we are examining one blackbox and we do not need another. In our example we can assume images as input feature, so the explanation might be heatmap with highlighted significant pixels \cite{Lapuschkin2015, Simonyan2014, Landecker2013}. If the our subject would be natural language processing the explanation might by highlighted significant text \cite{Arras2017, Li2016}.

We often express explanations as original feature vector with some kind of relevance score vector e.g. \ real-valued vector of same dimension as input where positive items indicates relevant features and zeros irrelevant \cite{Montavon2018}.

Speaking about human readability and understand ability. Different point of view on it quality is to look at the available time which the user (human) is available or allowed to spend on understanding the explanation \cite{Guidotti2018}. Then we can check the complexity or quality of the explanation by measuring the time to understand the explanation and if its average is above required treshold. Of course this we can observe under the assumption we know (and can check) level of the background knowledge of person performing the interpretation. And this leads us to next complication \cite{Guidotti2018}.

\citet{Montavon2018} address the main bias in the term \emph{interpretability} and state that one of problems we face is that the majority of the model evaluation metrics works with the model as with blackbox - sufficient information for evaluation is just set of predicted labels (or class probabilities) and ground-truth labels. Often, we do not observe model parameters, hyperparameters, model hypthesis space \dots. 

Another problem is that there are situations which are not easily trasferable to numerical form e.g. \ real-valued vector. Examples we can find in ethics or legality. Interpretability of such concepts is not ambigous even in the world without machine learning.


Based on \cite{Guidotti2018} and \cite{Lipton2016}, there are several aspects according to which we can categorize model explaining - knowledge of the model, scope of explanation and technique of explanation.

If we know the model we are explaning and we can investigate its functioning, parameters and other details during learning process we speak about white box explaining - often called \emph{transparency}. In this case the goal of explanation might by to answer the question \emph{how does the model work?}. On the other hand if we have only the output of the model without knowing what model is used and how it works we speak about black box explanaining - often called \emph{post-hoc} explanation. In this case the question is \emph{what else can the model tell us?}.

\emph{Global interpretability} means that we are able to intepret all predictions - we know intepretation $$f$$: $$\forall y \in \mathcal{Y}$$, $$\exists i \in \mathcal{D}$$ $$f: y\mapsto i$$. In another words the intepretation is mapping (or relation) and if the model is globaly interpretable this mapping is \emph{serial} (also called \emph{left-total}). On the other hand \emph{local interpretability} means that we are able to intepret only some so mentioned relation is not \emph{serial}.

Another aspect which is often considered is the fact if we explain single examples or set of examples. In case of explaining single example we might observe specific reason of particular prediction. If we explain batch of examples there is a space for generalization across the examples and avoid some noise in particular explanations.


\section{Explanability desiderata}
In \cite{Lipton2016} authors present very comprehensive, critical and real insight to desiderata of intepretability reasearch and an interpretable model properties. We list some of them.

\subsection{Trust}
This is first term which is very complicated due its own interpretation. We might build the trust on the performance of the model, so the better the model is the more trustworthy and maybe more interpretable it is. But in such case we just need high performance models and we have no questions. The situation is more complex in sense that we need to examine whole context. By context we mean the accuracy on specific examples e.g. \ unseen examples or examples where people are able to classify with high precision. The overall trust might be based on the model flexibility in different situations and what is more important that we should care about the trust evaluation repeatedly.

\subsection{Causality X Correlation}
Correlation of two random variables $X,Y$ is defined as $\rho_{X,Y}=\frac{cov(X,Y)}{\sigma_X\sigma_Y}$, where $cov(X,Y)$ denotes covariance and $\sigma$ standard deviation. On the other hand causality is defined as relation between $X$ (a cause) which contributes to the production of $Y$ (an effect), $X,Y$ might be event, process, state, object or generally random variables.

Every statistician was instructed about situations where the data shown us results we want to see. This is demonstrated on examples which might be so absurd that nobody should take it seriously. For example divorce rate in Maine correlates with per capita consumption of margarine between 2000 and 2009 \footnote{http://www.tylervigen.com/spurious-correlations}.

However, we have to remember that the underlying process generating the data for machine learning algorithms is assumed to be random no matter how complex it might be. We must not forget about the fact that the researcher is making assumptions and choosing what data are modelled. The statistical model itself should not serve as an argument for cause and effect relation between modelled variables. If we want make conclusion about causality we should involve other experiments and research in the particular domain to uncover the generating process itself. Correlation is easily measurable and conclusion about it is based only on its calculation. More on this topic can be found in \cite{Kenny1979}.

Even if we observe correlation and we know that the relation is not random it is difficult to make conclusion about the cause and effect. One of reason might by an confounding variable which is the variable influencing both features and states of our model. If there exist such variable we might falsely make a conclusion about causality although the correlation is caused by the confounder. More on this topic can be found in \cite{Skelly2012}.

\subsection{Transferability}
This problem addresses the situation that in usual setup we split our data randomly and create training and testing set. Then we observe generalization error estimate as difference between training and testing error. However, regarding the model deployment we should observe its behavior in real world as it might face different situations or even worse its deployment might influence the domain where deployed. The difference of training and deployment setup might be caused by strong assumptions we make but cannot meet because the data we have does not follow them strictly. This trend refers to robust statistic field where we mainly face the problem of the assumption violation \cite{Erceg-Hurn2008}.

\subsection{Informativness}
This point is about the ability of the model to extend human intuition and knowledge by pointing out the most important parts of comprehensive outputs. It casn also provide stronger overview on the space we are examining - provide some similarity measure on our examples. This might be very challenging task in the process of gaining labelled data using unsupervised or semi-supervised learning.

\subsection{Fair and ethical decision making}
If we want algorithms making autonomous decisions under our control and being of our interest we need to interpret its decisions. This very significant problem because we need to deal with the fact that the artificial inteligence is much more capable to make fast and precise decisions than humam. If we need to interpret it and control it we need to find mapping from this super fast and often unintuitive world to the world of our understanding. It is not clear if we do not degrade the capabilities of AI by trying to understand. However it is necessary because we need to avoid situations like Cambridge analytica affair during presidential ellections \cite{Boldyreva2018}. As we stated by adopting GDPR we face this challenge even for legal reasons. The field of Ethics in AI \cite{Siau2020} is inexhaustible well of challenges beginning by autonomous driving and decision of car and ending with absolute manipulation of a mass of people.

% \todo{list Desiderata of Interpretability Research and summarize each and try to connect it to cyber security (examples, especially in case of huge data, huge entrophy and causality X Correlation tradeoff)}
% \todo{make up some images, for example demostration of confounding variable}

% ○ How to detect that we are predictiong based on some confounding variable
% philozophy - Causality X Correlation - cite some literature
% The book of why, Judea Pearl


% \cite{Lipton2016}
% 4.4. Post-hoc interpretations can potentially mislead
% We caution against blindly embracing post-hoc notions of
% interpretability, especially when optimized to placate subjective demands. In such cases, one might - deliberately or
% not - optimize an algorithm to present misleading but plausible explanations. As humans, we are known to engage in
% this behavior, as evidenced in hiring practices and college
% admissions. Several journalists and social scientists have
% demonstrated that acceptance decisions attributed to virtues
% like leadership or originality often disguise racial or gender
% discrimination (Mounk, 2014). In the rush to gain acceptance for machine learning and to emulate human intelligence, we should be careful not to reproduce pathological
% behavior at scale.

% \cite{Guidotti2018} \todo{nice-to-have are another desiderata below}
% Desiderata of an Interpretable Model
% An interpretable model is required to provide an explanation. Thus, to realize an interpretable
% model, it is necessary to take into account the following list of desiderata, which are mentioned
% by a set of papers in the state of art [5, 28, 32, 45]:
% • Interpretability: to which extent the model and/or its predictions are human understandable. The most addressed discussion is related to how the interpretability can be measured.
% In Reference [32] a component for measuring the interpretability is the complexity of the
% predictive model in terms of the model size. According to the literature, we refer to interpretability also with the name comprehensibility.
% • Accuracy: to which extent the model accurately predicts unseen instances. The accuracy
% of a model can be measured using evaluation measures like the accuracy score, the F1-
% score [118], and so on. Producing an interpretable model maintaining competitive levels of
% accuracy is the most common target among the papers in the literature.
% • Fidelity: to which extent the model is able to accurately imitate a black-box predictor. The
% fidelity captures how much is good an interpretable model in the mimic of the behavior of
% a black-box. Similarly to the accuracy, the fidelity is measured in terms of accuracy score,
% F1-score, and so on, but with respect to the outcome of the black box.



% Nice-to-have:
% \cite{Mittelstadt2019}
% section about Human explanations are selective - just to philozophy it a little bit

\section{Interpretable model}
As we mentioned we can distinguish two basic types of explanation according to our goal - \emph{transparency} and \emph{post-hoc}.

\subsection{Transparency}
Authors in \cite{Lipton2016} reffer to several attributes which can by treated during a particular model's \emph{transparency} research - \emph{simulatability, decomposability} and \emph{algorithmic transparency}.

By \emph{simulatability} is meant that the model prediction can be simulated by a human in reasonable amount of time given the model parameters and input example. Obviously this capability is closely connected with the model complexity. This might seem like it is about the type of the model such that we can say that decision trees are more interpretable than neural nets. The truth is that simpler models like linear regression or decision trees tends to be more interpretable but it is because they are usually involved in straightforward use cases. On the other hand the simulatability is strictly determined by limited amount of human conginition which leads us to the conclusion that very complex decision tree is not more interpretable than a lightweight neural net.

\emph{Decomposability} stands for the the ability that each input, parameter and calculation admits an intuitive explanation. The input interpretability throw out of the game majority of models where dimensionality reduction and other feature engineering which inlude complex mapping is involved. The intepretation of parameters and calculation is for example that parameters of linear model are representing strenghts of association of variables or an human-readable description of decision tree nodes. An oposite might be a huge amount of weight and biases in neural net.

Last notion of \emph{algorithmic transparency} is about observing the learning algorithm and its mathematical background. The algorithm have to be fully explorable using mathematical tools for example in case of linear model the shape of error surface can be understood and we can prove that training will converge to unique solution. On the other hand heuristic algorithms used in deep models like \emph{stochastic gradient descent} can not be fully observed and we are cannot be sure about its adaptation in new situation.

Examples of model with significant level of transparency are Linear/Logistic reggression, decision trees, KNN, Rule based learners, generative additive models and bayesian models. More on this topic can be found in \cite{Arrieta2019}.


\subsection{Post-hoc}
Post-hoc interpretability has different goal than the previous category. It can extract more information from the model which might help to gain new overall knowledge or just understand why it predicts such a class for particular examples. This technique can be used to intepret opaque (not trasparent) models without examining their complex logic.

There are several techniques for post-hoc interpretation, their list follows.

\begin{itemize}
    \item \emph{Text explanation} - We still assume that we are not able to assign textual interpretation by hand because that might mean that our model is transparent. However, we might automate the inference of a textual form. We can train another model which maps the prediction of original model to its textual explanation. Example of such approach is in \cite{Krening2017}, where authors trained a reiforcement model to perform particular task and second model to explain decisions.
    \item \emph{Visualization} - Images and other visual outputs are considered to be very straightforward for human understanding. We often involve dimensionality reduction and other techniques to be able to display the situation in two or three dimensions such as \cite{Pearson1901}. Example of visualization of neural net explanation using heatmap can be found in \cite{Zeiler2013}.
    \item \emph{Local explanations} - This approach is about explaining specific parts (e.g. \ specific samples). It might find parts of sample which contribute to the predicition the most. An example might be saliency map usage for neural nets \cite{Simonyan2014}. It is important to emphasize that the explanation have to be treated in specific context e.g. \ saliency map might change drastically even if the example was changed only a little bit.
    \item \emph{Explanations by example} - Usually, if the teacher explain the theory on running example, especially in mathematics, there exists much greater chance that a lot of student would understand. The principle is the same in model explanation by example. The model can provide such explanation along with prediction i.e. \ prediction and set of examples which are in the same class  \cite{Caruana1999}. We might reveal something about the space in which examples are treated by the model - clusters based on predicted classes.
    \item \emph{Explanations by simplification} - As simpler models are usually easily interpretable this approach is aiming at model simplification while maintaining its performance.
    \item \emph{Feature relevance explanation} - We aim at scoring the input variables. Variables are later compared based on the their score and we can make conclusion which variables are the most important for particular prediction.
\end{itemize}

This kind of explaining is used with SVM, where we can see model simplification or local explanations. Another significant modelsa re neural nets and their variations where we see feature relevance and vizualization techniques. Examples can be found in \cite{Arrieta2019}.

Another example of model explaining is in \cite{Alvarez-Melis2018}, where authors focus self-explaining models. Other sources might be \cite{Strumbelj2013,Robnik-Sikonja2008,Montavon2018}.


\section{Explaining HMill models}
Our model is capable of modelling trees structured data - \emph{JSON} documents as stated in previous chapter. In this section we describe explanation method for \emph{HMill} models proposed and demonstrated in \cite{Pevny2020}. Introduced explainer attempts to explain complex structured \emph{HMill} model of from neural nets. It uses post-hoc explanation approach with \emph{feature relevance explanation}. So fat it is the only known approach of \emph{HMill} model explanation and as we do not aspire to find new solutions in this part we will follow author's results and choose method with the best results.

The goal of \emph{HMill} explainer is to find minimal subset of input sample (\emph{JSON} subtree) that is still classified to the same class as the original sample. Doing so we can identify possible bias e.g. \ in \cite{Pevny2020} authors concluded that their classifier is using timestamp field in \emph{JSON} sample the most, even if it was irrelevant information. It might also improve our understanding of domain specific knowledge as authors state.

In following text we assume \emph{HMill} binary classification model as blackbox decomposed only to two function $h$ and $f$. The first function is $h(d)=v$ where $d$ is \emph{JSON} document or its subtree and the output $v\in\mathbb{R}^{m}$ represents the embedded sample (all abstract model nodes are evaluated but the root). The embedded sample is than classified by $f(h(d))=c$, where $f$ denotes final abstract model node evaluation function and $c$ confidence that $d$ belongs to positive class.

\subsection{Explainer steps}
The basic idea builds on top of sub-tree selection problem solution. Our problem is a specific case of subtree selection. It can be formulated such that for given tree $T$, expensive evaluation function $r(t)=q$ ($q\in\mathbb{R}$) and treshold $\tau\in\mathbb{R}$ we want to find subtree $t$ with minimal number of nodes such that $r(t)\grq\tau$. 

In case of \emph{HMill} explainer the input tree is \emph{JSON} document and evaluation function is $f$ which outputs the confidence that document belongs to positive class. Authors introduced several sub-set selection methods which is less general case of subtree problem. It is not possible to adopt subset selection methods in subtree problems directly. Firtly, we have to decide if we need to maintain the tree structure of result or not. Authors mention two solutions to deal with this problem - one ignoring the structure and second exploiting it.

\subsection{Subset selection}
\paragraph{Greedy addition} starts with empty subset and in each steps adds new item until the treshold is achieved. Each new item is chosed such that the gain in the evaluation function is maximal across all elements which are not in the subset.
\paragraph{Heuristic addition} sorts the elements in the set by heuristic ranking (see below) and adds to the subset starting from the items with largest ranking until the treshold is achieved.
\paragraph{Random removal} might follow any of previous methods. It starts with initial subset which already achieved the treshold. It permutes all items in subset and removes items from the beggining until the evaluation drops below the treshold. If it drops lastly removed item is added back and algorithm continue with new permutation.
% \paragraph{Fine tuning}
\subsection{Subtree adoptation}
\paragraph{Flat search} performs subset selection on isolated nodes of the tree with only assumption that root is defaultly added to explanation. After subset selection is done all nodes which are not reachable from the root (there is no path from root to that particular node) it is removed from explanation. This can be done because they do not have any impact on classification based on the definition of \emph{HMill} model tree structure.
\paragraph{Level-by-leve search} performs subset selection on each level of the tree. On each level it is taking into account only nodes whose parents are in the explanation so far.

% Finding an explanation of \emph{HMill} model has two phases - subtree ranking and subtree selection. In the first phase all subtrees are heuristically ranked to reflect their importance for the final classification. Secondly, subtrees of a sample are searched through and evaluated by the model to find minimal explanation. There are multiple approaches adopted by authors for both phases. 

\subsection{Subtree ranking}
\subsubsection{Model gradient ranking}
This approach is based on high absolute value of gradients for most significant parts of input. That is adopted even in case of saliency map of image processing neural net. The crucial idea is to compute the gradient $gr=\frac{\partial f(h(d))}{\partial h(c)} \in \mathbb{R}^m$ of the model with respect to the embedding of particular subtree $c$ of particular sample $d$. Alternatively formulated, we want to examine how much a small change of embedding of a specific subtree influence model's prediction. Calculation of gradient ranking $r$ of a particular subtree follows: $r(c)=|\sum_{i=1}^{m}gr_i|$.

\subsubsection{GNN explainer mask ranking}
This method is originally designed to explain \emph{GNN} models by \cite{Ying2019}. The main idea is to use this explainer for the graph which is created by \emph{JSON} subtrees and edges between them. Explainer uses a mask $m=\mathbb{R}^{|\mathcal{E}|}$, where $|\mathcal{E}|$ denotes all edges between the subtrees and predecessors in \emph{JSON} document and $m_i\in\[0,1\]$. This mask represent how much information is passed along each edge (during update step of GNNs). If explainer is asked to provide an explanation of a classification decision on a particular subtree the mask is optimized using stochastic gradient descent to maximize the probability of correct classification on that subtree \cite{Pevny2020}. After the optimization is performed explainer suggests $k$ edges with largest values as explanation. The value of $k$ should be properly tuned because the explanation does not have to be classified to the same class as original sample for in advanced fixed $k$. The values of mask might be used as ranking in \emph{HMill} explainer.
\subsubsection{Banzhaf values}
Game theoretical approaches in feature extraction which uses \emph{Shapley values} and \emph{Banhaf values} can be found in \cite{Afghah2018}. Both comes form cooperative game theory. They originally express how much a certain player contributes to various coalitions on average. We can use them for various decisions like if the player should cooperate or not. In feature extraction Banzaf value might be used to describe interdependency among the extracted features and their relevancy to the target class. On the other hand \emph{shapley value} might be used to show the contribution of a particular feature in impoving accuracy of classification when all possible coalitions of features are considered. In case of \emph{HMill} explainer authors used both variants and they did not show significantly different results. 

A sampling algorithm is used to approximate the banzaf values \cite{Bachrach2010}. There are two values stored for each subtree in \emph{JSON} document- average confidence of the classifier in coalitions that include the subtree and the average confidence of coalitions that do not. Coalitions are generated randomly. After running a larger number of iterations, the approximation of Banzhaf values for each subtree is difference the two values stored in it. Banzof values approximation is used as subtree ranking. If we do not fix the seed the explanation is stochastic and might be instable regarding the number of iterations and the output.

% Clustering and explanation of sample and batch - nice-to-have

\subsection{Prior}
Authors in \cite{Pevny2020} present qualitative and quantitative analysis for differen \emph{HMill} explainer setup. The best result regarding the computational time and size of explanation were reported for Banzhaf values based heuristic addition approach in combination with Level-by-level search.


\section{Other methods for structured data}
We mentioned three alternative techniques for \emph{JSON} modelling in previous chapters - rules, flattening, graph neural nets. Flattening is technique of feature engineering more than modelling but rules might be considered modelling tool. Rules are usually transparent so their intepretability is straightforward. Examples of graph neural nets explaining can be found in \cite{Ying2019, Huang2020} 

The first part of thesis addressed the most important theoretical background of our method. It covers basics of malware analysis, machine learning (especially \emph{hmill} framework) and model explaining with \emph{hmill} Explainer. In following part we will discuss our setup, results and discussion.


% %%------------------------------------------------------------------------------------

% \todo{Summarize MillExplainer - \cite{Pevny2020}}
% Two main steps
%     sub-trees in a sample are heuristically ranked to reflect their importance for the final classification
%     subtrees of a sample are searched throughand evaluated by the model in order to find the minimalexplanation
% Summarize mentioned ways
% Summarize conclusions and results
%     Conclusion The best way are Banzhaf values

% address somehow the significance of confidence of neural net and what does it mean

% \todo{could be good to formulate pseudocode of the explainer also}

% Our target is explanation of \emph{HMill} model because it is also model we are about to use. In this section we will introduce solution proposed in \cite{Pevny2020} and we will reference other solutions which were presented on models capable consuming \emph{JSON} documents.

% If we do not specify we assume binary classification setup.

% As mentioned earlier we assume two main models capable processing tree-structured data. Those are \emph{Graph Neural Nets} and \emph{HMill}. Both methods described below are based on indentifying subset of original feature vector set (graph, tree-structured document) which has crucial role during prediction itself (in our case classification). \todo{connect to the theory above}



% Alternative model for structured data if 
% \todo{First shortly about gnn explainer and their results - https://cs.stanford.edu/people/jure/pubs/gnnexplainer-neurips19.pdf - \cite{Ying2019}, https://arxiv.org/pdf/2001.06216.pdf \cite{Huang2020}- adaptation of LIME in graph neural networks (which also can solve our problem theoretically)}


% If we need inspiration for this chapter - here https://arxiv.org/pdf/1910.10045v1.pdf

% \section{Explanation techniques}
% \citet{Guidotti2018} further categorize explanation techniques into several groups based on the basic princeple:
% \todo{summarize the categories below, and we should also reference it in hmill explaining section also, what we use}
% We further categorize in the subsections the various methods with respect to the type of interpretable explanator:
% • Decision Tree (DT) or Single Tree. It is commonly recognized that decision tree is one of
% the more interpretable and easily understandable models, primarily for global, but also for
% local, explanations. Indeed, a very widespread technique for opening the black box is the
% so-called “single-tree approximation.”
% • Decion Rules (DR) or Rule Based Explanator. Decision rules are among the more human understandable techniques. There exist various types of rules (illustrated in Section 3.3). They
% are used to explain the model, the outcome and also for the transparent design. We remark
% the existence of techniques for transforming a tree into a set of rules.
% • Features Importance (FI). A very simple but effective solution acting as either global or local
% explanation consists in returning as explanation the weight and magnitude of the features
% used by the black box. Generally the feature importance is provided by using the values of
% the coefficients of linear models used as interpretable models.
% • Saliency Mask (SM). An efficient way of pointing out what causes a certain outcome, especially when images or texts are treated, consists in using “masks” visually highlighting the
% determining aspects of the record analyzed. They are generally used to explain deep neural
% networks and can be viewed as a visual representation of FI.
% • Sensitivity Analysis (SA). It consists of evaluating the uncertainty in the outcome of a black
% box with respect to different sources of uncertainty in its inputs. It is generally used to
% develop visual tools for model inspection.
% • Partial Dependence Plot (PDP). These plots help in visualizing and understanding the relationship between the outcome of a black box and the input in a reduced feature space.
% • Prototype Selection (PS). This explanator consists in returning, together with the outcome,
% an example very similar to the classified record, to make clear which criteria the prediction
% was returned. A prototype is an object that is representative of a set of similar instances and
% is part of the observed points, or it is an artifact summarizing a subset of them with similar
% characteristics.
% • Activation Maximization (AM). The inspection of neural networks and deep neural network
% can be carried out also by observing which are the fundamental neurons activated with respect to particular input records, i.e., to look for input patterns that maximize the activation
% of a certain neuron in a certain layer. AM can be viewed also as the generation of an input
% image that maximizes the output activation (also called adversarial generation).



% \cite{Guidotti2018} \todo{check consistency of my definition and theirs, maybe extend my to match and cite}
% Definition 4.1 (Model Explanation Problem). Given a black box predictor b and a set of instances
% X, the model explanation problem consists in finding an explanation E ∈ E, belonging to a humaninterpretable domain E, through an interpretable global predictor cд = f (b,X) derived from the
% black box b and the instances X using some process f (·, ·). An explanation E ∈ E is obtained
% through cд, if E = εд (cд,X) for some explanation logic εд (·, ·), which reasons over cд and X.

% Notes:
% Two main phases I think which are model explaing using some method and second is intepretation - human understandable
% post-hoc intepretability X directly model interpretability
% Explaining model X explaining particular predictions
% local X global explanation
% correlation X causality
% definitions - basic terms (like for instance concept, math formalism...)

% Previous connection
% - Hmill models and their explanation
% Way through this chapter
% - Explaining in general
% - Explaining mill and hmill
% - Prior
% Next connection
% - Next should be little summary of first part of thesis
