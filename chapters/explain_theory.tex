\chapter{Model explaining}
In case of complex model structures like artificial neural nets we often want to explain its predictions. For example in case of malware we may be sure that the information used by the model is quite simple and straight-forward compared to the whole sample or at least that is in our expectation. That is why we often try to explain the models results to see if the explanation (e.g. \ subset of original sample which gives the model majority of information based on which it classifies) is hand in hand with our intuition. It could also be kind of hand brake to see that the model is not generalizing based on something relevant. For example in base prior article for us authors identified that their model is classifying mainly according to timestamp field in original sample, which was different for malware and for cleanware \cite{Pevny2020}. This was obviously mistake because this detail is not the difference between malware and cleanware.

Going from popular quote: \say{All models are wrong but some of them are useful} \todo{add name George E.P. Box} We should ask: \emph{What kind of model is useful?}
One of the greatest advice in machine learning and statistical data analysis is that less is more. If we are modeling some unknown process based on given data we should start from the simplest model and incrementaly proceed to more complex. This fact is derived from \emph{Bias x Variance tradeoff} \todo{Maybe some citation} and we can see practical consequences during model extrapolation but we can see even interpretability and explainability. That is why our next question should be: \emph{Why should we trust the model?} \cite{Ribeiro2016}.

A lot of articles like \cite{Zhang2016} leads us to conclusion that neural net generalization is against our intuition or expectations. Let us introduce basic facts about model explanation and later about \emph{HMill} explaining.

\todo{in some part (not necessarily here) we can mention common mistake distinguishing correlation and causality}

\section{General approaches}

Two main phases I think which are model explaing using some method and second is intepretation - human understandable

post-hoc intepretability X directly model interpretability
Explaining model X explaining particular predictions
local X global explanation

\cite{Montavon2018}
Techniques of interpretation are also becoming increasingly popular as a tool for exploration and analysis in the sciences. In combination with deep nonlinear machine learning models, they have been able to extract new insights from complex physical, chemical, or biological systems - our motivation is also to explore some new insight

Definition 1. An interpretation is the mapping of an abstract con-cept (e.g. a predicted class) into a domain that the human can make sense of.Examples of domains that are interpretable are images (arrays of pixels), or texts (sequences of words). A human can look at them and read them respectively. 

Examples of domains that are notinterpretable are abstract vector spaces (e.g. word embeddings [45]), or domains composed of undocumented input features (e.g. sequences with unknown words or symbols).

Definition 2. An explanation is the collection of features of the interpretable domain, that have contributed for a given example to produce a decision (e.g. classification or regression).
The features that form the explanation can be supplemented by relevance scores indicating to what extent each feature contributes. Practically, the explanation will be a real-valued vector of same size as the input, where relevant features are given positive scores, and irrelevant features are set to zero.An explanation can be, for example, a heatmap highlighting which pixels of the input image most strongly support the clas-sification decision [60,34,5]. In natural language processing, expla-nations can take the form of highlighted text [42,3]

\cite{Lipton2016}
\todo{just reference discussion about not clear terms like trust, interpretability, explanations...}
Consider that most common evaluation metrics for supervised learning require only predictions, together with
ground truth, to produce a score. These metrics can be
be assessed for every supervised learning model. So the
very desire for an interpretation suggests that in some scenarios, predictions alone and metrics calculated on these
predictions do not suffice to characterize the model (Figure
1). 
My: Sometimes the class probability distribution output is used

We turn now to consider the techniques and model properties that are proposed either to enable or to comprise interpretations. These broadly fall into two categories. The first
relates to transparency, i.e., how does the model work? The
second consists of post-hoc explanations, i.e., what else
can the model tell me? 

\todo{summarize section properties of interpretable model, our model is post-hoc local I think so we can say something more about that}


4.4. Post-hoc interpretations can potentially mislead
We caution against blindly embracing post-hoc notions of
interpretability, especially when optimized to placate subjective demands. In such cases, one might - deliberately or
not - optimize an algorithm to present misleading but plausible explanations. As humans, we are known to engage in
this behavior, as evidenced in hiring practices and college
admissions. Several journalists and social scientists have
demonstrated that acceptance decisions attributed to virtues
like leadership or originality often disguise racial or gender
discrimination (Mounk, 2014). In the rush to gain acceptance for machine learning and to emulate human intelligence, we should be careful not to reproduce pathological
behavior at scale.

\cite{Guidotti2018}

Black box X White box

Global and Local Interpretability: A model may be completely interpretable, i.e., we are able to
understand the whole logic of a model and follow the entire reasoning leading to all the different
possible outcomes. In this case, we are speaking about global interpretability. Instead, we indicate
with local interpretability the situation in which it is possible to understand only the reasons for
a specific decision: only the single prediction/decision is interpretable.

Time Limitation: An important aspect is the time that the user is available or is allowed to spend
on understanding an explanation. The user time availability is strictly related to the scenario where
the predictive model has to be used. Therefore, in some contexts where the user needs to quickly
take the decision (e.g., a disaster is imminent), it is preferable to have an explanation simple to
understand. While in contexts where the decision time is not a constraint (e.g., during a procedure
to release a loan) one might prefer a more complex and exhaustive explanation.

Nature of User Expertise: Users of a predictive model may have different background knowledge
and experience in the task: decision-makers, scientists, compliance and safety engineers, data scientists, and so on. Knowing the user experience in the task is a key aspect of the perception of
interpretability of a model. Domain experts may prefer a larger and more sophisticated model over
a smaller and sometimes more opaque one.

The European Parliament recently adopted the General Data Protection Regulation (GDPR),
which has become law in May 2018. An innovative aspect of the GDPR are the clauses on automated decision-making, including profiling, which for the first time introduce, to some extent,
a right of explanation for all individuals to obtain “meaningful explanations of the logic involved”
when automated decision making takes place. Despite divergent opinions among legal scholars
regarding the real scope of these clauses [36, 74, 126], there is a general agreement on the need for
the implementation of such a principle is urgent and that it represents today a huge open scientific
challenge. Without an enabling technology capable of explaining the logic of black boxes, the right
to an explanation will remain a “dead letter.”

Desiderata of an Interpretable Model
An interpretable model is required to provide an explanation. Thus, to realize an interpretable
model, it is necessary to take into account the following list of desiderata, which are mentioned
by a set of papers in the state of art [5, 28, 32, 45]:
• Interpretability: to which extent the model and/or its predictions are human understandable. The most addressed discussion is related to how the interpretability can be measured.
In Reference [32] a component for measuring the interpretability is the complexity of the
predictive model in terms of the model size. According to the literature, we refer to interpretability also with the name comprehensibility.
• Accuracy: to which extent the model accurately predicts unseen instances. The accuracy
of a model can be measured using evaluation measures like the accuracy score, the F1-
score [118], and so on. Producing an interpretable model maintaining competitive levels of
accuracy is the most common target among the papers in the literature.
• Fidelity: to which extent the model is able to accurately imitate a black-box predictor. The
fidelity captures how much is good an interpretable model in the mimic of the behavior of
a black-box. Similarly to the accuracy, the fidelity is measured in terms of accuracy score,
F1-score, and so on, but with respect to the outcome of the black box.

Definition 4.1 (Model Explanation Problem). Given a black box predictor b and a set of instances
X, the model explanation problem consists in finding an explanation E ∈ E, belonging to a humaninterpretable domain E, through an interpretable global predictor cд = f (b,X) derived from the
black box b and the instances X using some process f (·, ·). An explanation E ∈ E is obtained
through cд, if E = εд (cд,X) for some explanation logic εд (·, ·), which reasons over cд and X.

We further categorize in the subsections the various methods with respect to the type of interpretable explanator:
• Decision Tree (DT) or Single Tree. It is commonly recognized that decision tree is one of
the more interpretable and easily understandable models, primarily for global, but also for
local, explanations. Indeed, a very widespread technique for opening the black box is the
so-called “single-tree approximation.”
• Decion Rules (DR) or Rule Based Explanator. Decision rules are among the more human understandable techniques. There exist various types of rules (illustrated in Section 3.3). They
are used to explain the model, the outcome and also for the transparent design. We remark
the existence of techniques for transforming a tree into a set of rules.
• Features Importance (FI). A very simple but effective solution acting as either global or local
explanation consists in returning as explanation the weight and magnitude of the features
used by the black box. Generally the feature importance is provided by using the values of
the coefficients of linear models used as interpretable models.
• Saliency Mask (SM). An efficient way of pointing out what causes a certain outcome, especially when images or texts are treated, consists in using “masks” visually highlighting the
determining aspects of the record analyzed. They are generally used to explain deep neural
networks and can be viewed as a visual representation of FI.
• Sensitivity Analysis (SA). It consists of evaluating the uncertainty in the outcome of a black
box with respect to different sources of uncertainty in its inputs. It is generally used to
develop visual tools for model inspection.
• Partial Dependence Plot (PDP). These plots help in visualizing and understanding the relationship between the outcome of a black box and the input in a reduced feature space.
• Prototype Selection (PS). This explanator consists in returning, together with the outcome,
an example very similar to the classified record, to make clear which criteria the prediction
was returned. A prototype is an object that is representative of a set of similar instances and
is part of the observed points, or it is an artifact summarizing a subset of them with similar
characteristics.
• Activation Maximization (AM). The inspection of neural networks and deep neural network
can be carried out also by observing which are the fundamental neurons activated with respect to particular input records, i.e., to look for input patterns that maximize the activation
of a certain neuron in a certain layer. AM can be viewed also as the generation of an input
image that maximizes the output activation (also called adversarial generation).

\cite{Mittelstadt2019}
Much recent work has been dedicated to rendering machine learning models interpretable or explainable. Two broad aims of work on
interpretability have been recognised in the literature: transparency
and post-hoc interpretation. Transparency addresses how a model
functions internally, whereas post-hoc interpretations concern how
the model behaves (Lepri et al., 2017; Lipton, 2016; Montavon et al.,
2017).

section about Human explanations are selective - just to philozophy it a little bit

general motivation

definitions - basic terms (like for instance concept, math formalism...)


prior
- each of papers presents their own approach wich could be mentioned
NN
    Motivation example - https://arxiv.org/pdf/1311.2901.pdf (quite often seen results )
https://arxiv.org/pdf/1806.07538.pdf
https://arxiv.org/pdf/1602.04938.pdf
https://link.springer.com/content/pdf/10.1007/s10115-013-0679-x.pdf
https://ieeexplore.ieee.org/stamp/stamp.jsp?tp=&arnumber=4407709
https://reader.elsevier.com/reader/sd/pii/S1051200417302385?token=E746E7B6AEA70FD3F88C8EDB1CE81407F166E7078D3011812BBC4D2DEB1597D112CCA6A919B515991126E136897B721C&originRegion=eu-west-1&originCreation=20210405200556 \cite{Lipton2016}



\section{Explaining HMill models}
Our target is explanation of \emph{HMill} model because it is also model we are about to use. In this section we will introduce solution proposed in \cite{Pevny2020} and we will reference other solutions which were presented on models capable consuming \emph{JSON} documents.

If we do not specify we assume binary classification setup.

As mentioned earlier we assume two main models capable processing tree-structured data. Those are \emph{Graph Neural Nets} and \emph{HMill}. Both methods described below are based on indentifying subset of original feature vector set (graph, tree-structured document) which has crucial role during prediction itself (in our case classification). \todo{connect to the theory above}

\todo{First shortly about gnn explainer and their results - https://cs.stanford.edu/people/jure/pubs/gnnexplainer-neurips19.pdf, https://arxiv.org/pdf/2001.06216.pdf - adaptation of LIME in graph neural networks (which also can solve our problem theoretically)}

\todo{Summarize MillExplainer - file:///C:/Users/domia/Downloads/Explainability_ICML2021.pdf}
Two main steps
    sub-trees in a sample are heuristically ranked to reflect their importance for the final classification
    subtrees of a sample are searched throughand evaluated by the model in order to find the minimalexplanation
Summarize mentioned ways
Summarize conclusions and results
    Conclusion The best way are Banzhaf values


\section{First part summary}
\todo{the section heading does not have to be presented}
Summarize main point in this part and motivate next part


%%------------------------------------------------------------------------------------


Previous connection
- Hmill models and their explanation
Way through this chapter
- Explaining in general
- Explaining mill and hmill
- Prior
Next connection
- Next should be little summary of first part of thesis
