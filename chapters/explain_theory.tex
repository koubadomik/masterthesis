\chapter{Model explaining} \label{chap:expth}
In following chapter we describe model explainability in general - definition, motivation, prior arts. Finally, we give details of \emph{hmill} explainer which we use in experiments.




In case of complex model structures like artificial neural nets we often want to explain its predictions. For example in case of malware we may be sure that the information used by the model is quite simple and straight-forward compared to the whole sample or at least that is in our expectation. That is why we often try to explain the models results to see if the explanation (e.g. \ subset of original sample which gives the model majority of information based on which it classifies) is hand in hand with our intuition. It could also be kind of hand brake to see that the model is not generalizing based on something relevant. For example in base prior article for us authors identified that their model is classifying mainly according to timestamp field in original sample, which was different for malware and for cleanware \cite{Pevny2020}. This was obviously mistake because this detail is not the difference between malware and cleanware.

Going from popular quote: \say{All models are wrong but some of them are useful} \todo{add name George E.P. Box} We should ask: \emph{What kind of model is useful?}
One of the greatest advice in machine learning and statistical data analysis is that less is more. If we are modeling some unknown process based on given data we should start from the simplest model and incrementaly proceed to more complex. This fact is derived from \emph{Bias x Variance tradeoff} \todo{Maybe some citation, https://blog.insightdatascience.com/always-start-with-a-stupid-model-no-exceptions-3a22314b9aaa, describe better} and we can see practical consequences during model extrapolation but we can see even interpretability and explainability. That is why our next question should be: \emph{Why should we trust the model?} \cite{Ribeiro2016}.


\todo{another good motivation}
The European Parliament recently adopted the General Data Protection Regulation (GDPR),
which has become law in May 2018. An innovative aspect of the GDPR are the clauses on automated decision-making, including profiling, which for the first time introduce, to some extent, a right of explanation for all individuals to obtain “meaningful explanations of the logic involved”
when automated decision making takes place. Despite divergent opinions among legal scholars
regarding the real scope of these clauses [36, 74, 126], there is a general agreement on the need for
the implementation of such a principle is urgent and that it represents today a huge open scientific
challenge. Without an enabling technology capable of explaining the logic of black boxes, the right
to an explanation will remain a “dead letter.” \cite{Guidotti2018}

A lot of articles like \cite{Zhang2016} leads us to conclusion that neural net generalization is against our intuition or expectations. Let us introduce basic facts about model 
explanation and later about \emph{HMill} explaining.

This field is a little bit like a border between artificial inteligence, philosophy and other branches. We do not aspire to cover everything, the goal is to summarize terminology and basic challenges. This part of the thesis is more experimental, just to see what results we have and how we can interpret them.

\section{General approaches}

Notes:
Two main phases I think which are model explaing using some method and second is intepretation - human understandable
post-hoc intepretability X directly model interpretability
Explaining model X explaining particular predictions
local X global explanation
correlation X causality
definitions - basic terms (like for instance concept, math formalism...)

\cite{Montavon2018}
Techniques of intepretation and explaining are growing in popularity as a tool for further statistical model analysis. It might lead to better model understanding or it can shed some more light on examined domain (extract new knowledge) \cite{Montavon2018}

Deriving to \cite{Montavon2018} we can define two basic terms.
\begin{definition}
An \emph{iterpretation} is the mapping $$\mathcal{Y}\rightarrow\mathcal{D}$$. $$\mathcal{Y}$$ was defined in \todo{ref chapter Classification}) as \emph{state space}, in general it can be any kind of abstract concept (real-valued vectors, sequences\dots).  $$\mathcal{D}$$ denotes domain which is human-readable and understandable (image, heatmap, sequence of words\dots). Typically $$D$$ is the same as feature space \todo{define feature space} $$\mathcal{X}$$ or its transformation.
\end{definition}

\begin{definition}
    An \emph{explanation} $$e \in \mathcal{d}$$ is a subset $$e$$ of a interpretable domain $$\mathcal{D}$$ that contributed for the predicted state $$y \in \mathcal{Y}$$ (or contributed significantly more than other members and so it is cause of the result).
\end{definition}

This definition is quite vague because human-readability and understadability is not something to measure or observe precisely. We might assume that the original \emph{state space} is not straight-forward for human understanding which is usually true e.g. \ neural net's predicted class (just 1 or 0). In case of a neural net we can see only its parameters i.e. \emph{weights} and \emph{biases}, input features and than prediction. But the question is \emph{Why?}. Fortunately, final layer (or funciton) in neural network is \emph{softmax}, so we can try to interpret class probabilites instead just 1 or 0 outputs. We might want to see the subset of input features which is responsible for the majority of accumulated class probability. This subset should be intepretable by human as mentioned in definition because we are examining one blackbox and we do not need another. In our example we can assume images as input feature, so the explanation might be heatmap with highlighted significant pixels \cite{Lapuschkin2015, Simonyan2014, Landecker2013}. If the our subject would be natural language processing the explanation might by highlighted significant text \cite{Arras2017, Li2016}.

We often express explanations as original feature vector with some kind of relevance score vector e.g. \ real-valued vector of same dimension as input where positive items indicates relevant features and zeros irrelevant \cite{Montavon2018}.

Speaking about human readability and understand ability. Different point of view on 'measuring' it is to look at the available time which the user (human) is available or allowed to spend on understanding the explanation \cite{Guidotti2018}. Then we can check the complexity or quality of the explanation by measuring the time to understand the explanation and if its average is above required treshold. Of course this we can observe under the assumption we know (and can check) level of the background knowledge of person performing the interpretation. And this leads us to next complication \cite{Guidotti2018}.

Nice-to-have:
\cite{Mittelstadt2019}
section about Human explanations are selective - just to philozophy it a little bit



\cite{Lipton2016}
These definitions are not so rigorous how we would like them to be. We would rather state that another source of information is more clear but it is not. However, in \cite{Lipton2016} authors present very comprehensive, critical and real insight to desiderata of intepretability reasearch and an interpretable model properties.


\citet{Montavon2018} address the main bias in the term \emph{interpretability} and state that one of problems we face is that the majority of the model evaluation metrics works with the model as with blackbox - sufficient information for evaluation is just set of predicted labels (or class probabilities) and ground-truth labels \todo{maybe define those two}. Often, we do not observe model parameters, hyperparameters, model hypthesis space \dots. 

Another problem is that there are situations which are not easily trasferable to numerical form e.g. \ real-valued vector. Examples we can find in ethics or legality. Interpretability of such concepts is not ambigous even in the world without machine learning.


\todo{list Desiderata of Interpretability Research and summarize each and try to connect it to cyber security (examples, especially in case of huge data, huge entrophy and causality X Correlation tradeoff)}
\todo{make up some images, for example demostration of confounding variable}

○ How to detect that we are predictiong based on some confounding variable (reference theory chapter we should have described it there)...
philozophy - Causality X Correlation - cite some literature
The book of why, Judea Pearl

\todo{add section about properties of interpretable model}

    We turn now to consider the techniques and model properties that are proposed either to enable or to comprise interpretations. These broadly fall into two categories. The first
    relates to transparency, i.e., how does the model work? The
    second consists of post-hoc explanations, i.e., what else
    can the model tell me? 

    \todo{summarize section properties of interpretable model, our model is post-hoc local I think so we can say something more about that}

    4.4. Post-hoc interpretations can potentially mislead
    We caution against blindly embracing post-hoc notions of
    interpretability, especially when optimized to placate subjective demands. In such cases, one might - deliberately or
    not - optimize an algorithm to present misleading but plausible explanations. As humans, we are known to engage in
    this behavior, as evidenced in hiring practices and college
    admissions. Several journalists and social scientists have
    demonstrated that acceptance decisions attributed to virtues
    like leadership or originality often disguise racial or gender
    discrimination (Mounk, 2014). In the rush to gain acceptance for machine learning and to emulate human intelligence, we should be careful not to reproduce pathological
    behavior at scale.

    we can also reference other authors in this field- \cite{Mittelstadt2019}
    Much recent work has been dedicated to rendering machine learning models interpretable or explainable. Two broad aims of work on
    interpretability have been recognised in the literature: transparency
    and post-hoc interpretation. Transparency addresses how a model
    functions internally, whereas post-hoc interpretations concern how
    the model behaves (Lepri et al., 2017; Lipton, 2016; Montavon et al.,
    2017).



If we know the model we are explaning and we can investigate its functioning, parameters and other details during learning process we speak about \emph{white box explaining}. On the other hand if we have only the output of the model without knowing what model is used and how it works we speak about \emph{black box explanaining}.

\cite{Guidotti2018} aims at blackbox models and their intepretability. Authors distinguish two points of view. - \emph{global} and \emph{local} intepretability of the model.

\emph{Global interpretability} means that we are able to intepret all predictions - we know intepretation $$f$$: $$\forall y \in \mathcal{Y}$$, $$\exists i \in \mathcal{D}$$ $$f: y\mapsto i$$. In another words the intepretation is mapping (or relation) and if the model is globaly interpretable this mapping is \emph{serial} (also called \emph{left-total}).

On the other hand \emph{local interpretability} means that we are able to intepret only some so mentioned relation is not \emph{serial}.


\cite{Guidotti2018} \todo{nice-to-have are another desiderata below}
Desiderata of an Interpretable Model
An interpretable model is required to provide an explanation. Thus, to realize an interpretable
model, it is necessary to take into account the following list of desiderata, which are mentioned
by a set of papers in the state of art [5, 28, 32, 45]:
• Interpretability: to which extent the model and/or its predictions are human understandable. The most addressed discussion is related to how the interpretability can be measured.
In Reference [32] a component for measuring the interpretability is the complexity of the
predictive model in terms of the model size. According to the literature, we refer to interpretability also with the name comprehensibility.
• Accuracy: to which extent the model accurately predicts unseen instances. The accuracy
of a model can be measured using evaluation measures like the accuracy score, the F1-
score [118], and so on. Producing an interpretable model maintaining competitive levels of
accuracy is the most common target among the papers in the literature.
• Fidelity: to which extent the model is able to accurately imitate a black-box predictor. The
fidelity captures how much is good an interpretable model in the mimic of the behavior of
a black-box. Similarly to the accuracy, the fidelity is measured in terms of accuracy score,
F1-score, and so on, but with respect to the outcome of the black box.

\cite{Guidotti2018} \todo{check consistency of my definition and theirs, maybe extend my to match and cite}
Definition 4.1 (Model Explanation Problem). Given a black box predictor b and a set of instances
X, the model explanation problem consists in finding an explanation E ∈ E, belonging to a humaninterpretable domain E, through an interpretable global predictor cд = f (b,X) derived from the
black box b and the instances X using some process f (·, ·). An explanation E ∈ E is obtained
through cд, if E = εд (cд,X) for some explanation logic εд (·, ·), which reasons over cд and X.


\citet{Guidotti2018} further categorize explanation techniques into several groups based on the basic princeple:
\todo{summarize the categories below, and we should also reference it in hmill explaining section also, what we use}
We further categorize in the subsections the various methods with respect to the type of interpretable explanator:
• Decision Tree (DT) or Single Tree. It is commonly recognized that decision tree is one of
the more interpretable and easily understandable models, primarily for global, but also for
local, explanations. Indeed, a very widespread technique for opening the black box is the
so-called “single-tree approximation.”
• Decion Rules (DR) or Rule Based Explanator. Decision rules are among the more human understandable techniques. There exist various types of rules (illustrated in Section 3.3). They
are used to explain the model, the outcome and also for the transparent design. We remark
the existence of techniques for transforming a tree into a set of rules.
• Features Importance (FI). A very simple but effective solution acting as either global or local
explanation consists in returning as explanation the weight and magnitude of the features
used by the black box. Generally the feature importance is provided by using the values of
the coefficients of linear models used as interpretable models.
• Saliency Mask (SM). An efficient way of pointing out what causes a certain outcome, especially when images or texts are treated, consists in using “masks” visually highlighting the
determining aspects of the record analyzed. They are generally used to explain deep neural
networks and can be viewed as a visual representation of FI.
• Sensitivity Analysis (SA). It consists of evaluating the uncertainty in the outcome of a black
box with respect to different sources of uncertainty in its inputs. It is generally used to
develop visual tools for model inspection.
• Partial Dependence Plot (PDP). These plots help in visualizing and understanding the relationship between the outcome of a black box and the input in a reduced feature space.
• Prototype Selection (PS). This explanator consists in returning, together with the outcome,
an example very similar to the classified record, to make clear which criteria the prediction
was returned. A prototype is an object that is representative of a set of similar instances and
is part of the observed points, or it is an artifact summarizing a subset of them with similar
characteristics.
• Activation Maximization (AM). The inspection of neural networks and deep neural network
can be carried out also by observing which are the fundamental neurons activated with respect to particular input records, i.e., to look for input patterns that maximize the activation
of a certain neuron in a certain layer. AM can be viewed also as the generation of an input
image that maximizes the output activation (also called adversarial generation).



prior
- each of papers presents their own approach wich could be mentioned
NN
    Motivation example - https://arxiv.org/pdf/1311.2901.pdf (quite often seen results ) \cite{Zeiler2013}
https://arxiv.org/pdf/1806.07538.pdf - \cite{Alvarez-Melis2018}
https://link.springer.com/content/pdf/10.1007/s10115-013-0679-x.pdf - \cite{Strumbelj2013}
https://ieeexplore.ieee.org/stamp/stamp.jsp?tp=&arnumber=4407709 \cite{Robnik-Sikonja2008}
https://reader.elsevier.com/reader/sd/pii/S1051200417302385?token=E746E7B6AEA70FD3F88C8EDB1CE81407F166E7078D3011812BBC4D2DEB1597D112CCA6A919B515991126E136897B721C&originRegion=eu-west-1&originCreation=20210405200556 \cite{Lipton2016}


\section{Explaining HMill models}

address somehow the significance of confidence of neural net and what does it mean

\todo{could be good to formulate pseudocode of the explainer also}

Our target is explanation of \emph{HMill} model because it is also model we are about to use. In this section we will introduce solution proposed in \cite{Pevny2020} and we will reference other solutions which were presented on models capable consuming \emph{JSON} documents.

If we do not specify we assume binary classification setup.

As mentioned earlier we assume two main models capable processing tree-structured data. Those are \emph{Graph Neural Nets} and \emph{HMill}. Both methods described below are based on indentifying subset of original feature vector set (graph, tree-structured document) which has crucial role during prediction itself (in our case classification). \todo{connect to the theory above}

\todo{First shortly about gnn explainer and their results - https://cs.stanford.edu/people/jure/pubs/gnnexplainer-neurips19.pdf - \cite{Ying2019}, https://arxiv.org/pdf/2001.06216.pdf \cite{Huang2020}- adaptation of LIME in graph neural networks (which also can solve our problem theoretically)}

\todo{Summarize MillExplainer - \cite{Pevny2020}}
Two main steps
    sub-trees in a sample are heuristically ranked to reflect their importance for the final classification
    subtrees of a sample are searched throughand evaluated by the model in order to find the minimalexplanation
Summarize mentioned ways
Summarize conclusions and results
    Conclusion The best way are Banzhaf values

Here or at the beginning of the section we should classify proposed solutions in context of the theory above. - our model is post-hoc local and quite complex in terms the data dimension..., also connect it to mentioned categories

\todo{idea: some kind of explanation is even more complex class hierarchy, I did not see the purpose in classifying something else than malware X cleanware, but the class hierarchy can give us some kind of clues}


The first part of thesis addressed the most important theoretical background of our method. It covers basics of malware analysis, machine learning (especially \emph{hmill} framework) and model explaining with \emph{hmill} Explainer. In following part we will discuss our setup, results and discussion.


%%------------------------------------------------------------------------------------


% Previous connection
% - Hmill models and their explanation
% Way through this chapter
% - Explaining in general
% - Explaining mill and hmill
% - Prior
% Next connection
% - Next should be little summary of first part of thesis
