\chapter{Hierarchical Multiple Instance learning}
The goal of this thesis is to use specifically hmill learning to classify malware samples. Using it we want to  build on the prior work in this area. \todo{add references}

In this chapter we describe Hmill framework and use cases where this kind of learning was involved. Then we will address its usage in malware classification and other applications in cyber security field.


Stand on Madlik chapter, his citations (be aware), somol and Pevny

Previous connection
- We focus on structured data and our tool is hmill, let's describe it
Way through this chapter
- multiple instance learning
- hierarchical mill
- prior
Next connection
- Not necessary, but can connect to next chapter slightly (we want to explain this kind of model)




- GOALS 
- Learn the hierarchical multiple instance learning framework (HMill)
- describe theory, connect it to classical approach
- Usual usage of this type of learning - classification, regression...
- Our usecase and experiments with this kind of learning in malware classification field (prior)