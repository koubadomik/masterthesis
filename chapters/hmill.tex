\chapter{Hierarchical Multiple Instance learning}
The goal of this thesis is to use specifically hierarchical multiple instance learning to classify malware samples. Using it we want to follow the prior work in this area \cite{Mandlik2020}.
In this chapter we describe multiple instance learning as extended approach compared to previus chapter. In the second part we will describe hierarchical multiple instance learning framework as modelling tool to model tree-structured data. Finally, we will address its usage in malware classification and other applications in cyber security field.

\section{Multiple instance learning - Mill}
At first let us describe and define problem of \emph{Multiple instance learning} its origin and formalism.

The most cited original article in this field is \citet{Dietterich1997}, it is not the very first formulation of such problem (it is in \cite{Keeler1991}) but fist usage of term \emph{Multiple instance learning}.

On figure \todo{reference figure} we can see basic idea behind \emph{Mill} problem. Standard learning described in chapter \todo{ref chapter} we can see as a special case of \emph{Mill}. Compared to defined formalism we have to change definition of \emph{example} \todo{ref section example in chapter 1}. 
\paragraph{Example}
Firstly we for our puprose we assume that we are in \emph{supervised} setting. Our examples consist of two parts - \emph{bag} $$b$$ and \emph{state} $$y$$. State was defined earlier and its meaning is the same but it is related to bag and sometimes called \emph{bag label}. \emph{Bag} $$b$$ is set of feature vectors $$x_i$$, these vectors are called \emph{instances} and usually lives in standard feature space $$x_i \in \mathbb{R}^{n}$$. Its length is $$|b| \in \mathbb{N}$$ and can be even zero. We assume that $$b \in \mathcal{B}$$ which denotes \emph{bag space}. \emph{Bag space} might also be seen as $$\mathcal{B} = \mathrm{Fin}(\mathcal{X})$$ which denotes all finite subsets of $$\mathcal{X}$$. Now we can see that standard learning is just \emph{Mill} problem where holds $$|b| = 1$$.

For the purposes of our thesis we assume mill classification problem so $$|\mathcal{Y}|$$ is finite. One more additional note is that some researches and even the original has one assumption (sometimes called \emph{Standard assumption}). This assumption is about underlying \emph{instance label} and their relation to \emph{bag labels}. This assumptions is often relaxed \cite{Xu2003} so even we do.

\todo{I have to mention some citations the theory I built on top of Mandlik and Amores!}

\todo{image of standard setting vs mill setting}

\todo{if we would like to have some fancy figure we can try something with keys}

The motivation example in original paper \cite{Dietterich1997} is formulated in following way. Imagine we have keys and some of them unlock specific door and some of them not (no other choice). Important is that we do not have access to the door. In standard learning is our goal to learn classification model which consumes key and outputs if it is able to open or not. But in \emph{multiple instance learning} setting we receive whole chains of several keys and without checking every single key we have to classify if the whole chain is able to open the door.

More on this topis we can find in \cite{Amores2013} where we can find formulated paradigms of solving such issues.

\todo{List paradigm and shortly describe and add some examples (prior) of solutions - cite}

\todo{some real world examples (not only pevny and Mandlik).}

\todo{Last solution describe neural net architecture - cite Somol, Pevny, Dedic (3 papers) and even Mandlik with the notation on page 14,15}
\todo{Mention Zaheer with deep sets}
\todo{Add own figure of neural net architecture}

- Prior experiments
    - Add something from another domain
    - https://www.etsmtl.ca/Unites-de-recherche/LIVIA/Recherche-et-innovation/Publications/Publications-2017/mil_marc_2017.pdf
    - The most impotant
        - https://link.springer.com/content/pdf/10.1007%2F978-3-319-59072-1.pdf (somol, pevny)
        - https://dl.acm.org/doi/pdf/10.1145/2996758.2996761
        - Stiborek - https://reader.elsevier.com/reader/sd/pii/S0957417417307170?token=492DD375F9C6F2836E0F6B394FAA2BC3E6663F3EFD26E24D7295DBFDCB870D23C3B120BCCC9731B0E6399BEFEE29C4A2

        - https://arxiv.org/pdf/2002.04059.pdf - Pevny, Dedic
        - https://arxiv.org/pdf/1906.00764.pdf - Pevny Kovarik, 
        - https://papers.nips.cc/paper/2017/file/f22e4747da1aa27e363d86d40ff442fe-Paper.pdf - same idea, but different name (just marginally)

\section{HMILL framework}
The idea of the neural net architecture by \todo{ref names Somol and Pevny} was followed by \todo{ref Mandlik and Pevny}. This led to formulation of \emph{HMill framework} in \cite{Mandlik2020} which refers to \emph{Hierarchical Multiple instance learning framework}. This framework builds on the top of the idea of hierarchically composed models. Inspite the fact that the hierarchical learning is generalization of standard learning we can cover quite complicated structured data (which is later used by us).

- Framework definition
    - go to Mandlik
- prior will be sufficient
    - Mandlik's results using the framework
    - https://arxiv.org/pdf/2002.04059.pdf

The conclusion of this chapter and maybe the most significant conslusion for our work is that it is possible to use \emph{HMill framework} to classify tree-structured data e.g. \ \emph{JSON} documents. \citeauthor{Mandlik2020} presents convincing results (for example on \emph{device identification} example) and that is why this framework is directly in assignment of this thesis.

Regardless of quite comprehensive theory which is covered in this part of the thesis, \emph{HMill} framework is still not everything. Our goal is quite cross-cutting and the model we are going to train we want also explain. What we mean by explanation we describe in next chapter.

%---------------------REMAINING-------------------------------------

HMILL:
- assumptions - https://en.wikipedia.org/wiki/Multiple_instance_learning (Be aware, I think those are not true all at once...), also algorithm list is presented, could be mentioned, and other interesting references like problem generalization
- between modern approaches mention adapting of single instance algorithms

Previous connection
- We focus on structured data and our tool is hmill, let's describe it
Way through this chapter
- multiple instance learning
- hierarchical mill
- prior
    - Prior in various fields
    - Prior in cybersec
Next connection
- Not necessary, but can connect to next chapter slightly (we want to explain this kind of model)

Stand on Madlik chapter, his citations (be aware), somol and Pevny


- GOALS 
- Learn the hierarchical multiple instance learning framework (HMill)
- describe theory, connect it to classical approach
- Usual usage of this type of learning - classification, regression...
- Our usecase and experiments with this kind of learning in malware classification field (prior)