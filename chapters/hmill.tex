\chapter{Hierarchical Multiple Instance learning}
The goal of this thesis is to use specifically hmill learning to classify malware samples. Using it we want to  build on the prior work in this area. \todo{add references}

In this chapter we describe Hmill framework and use cases where this kind of learning was involved. Then we will address its usage in malware classification and other applications in cyber security field.

\section{Multiple instance learning}
At first let us describe and define problem of \emph{Multiple instance learning} its origin and formalism.
- https://www.ics.uci.edu/~rickl/publications/1997-aij.pdf
- History - shortly https://proceedings.neurips.cc/paper/1990/file/e46de7e1bcaaced9a54f1e9d0d2f800d-Paper.pdf (first mentions)
- Definition, simple example and formalism
    - https://www.ics.uci.edu/~rickl/publications/1997-aij.pdf - First term appearance, problem definition, example
    - assumptions - https://en.wikipedia.org/wiki/Multiple_instance_learning (Be aware, I think those are not true all at once...), also algorithm list is presented, could be mentioned, and other interesting references like problem generalization
        - between modern approaches mention adapting of single instance algorithms
    - Solutions paradigms, problem examples - https://reader.elsevier.com/reader/sd/pii/S0004370213000581?token=CFE39482CE50EAB37819E58C916355FFEB127B223C806C52A7FB11B383542ABFC678FDA51CE6772A6812FD9366C86666
- Prior experiments
    - Add something from another domain
    - https://www.etsmtl.ca/Unites-de-recherche/LIVIA/Recherche-et-innovation/Publications/Publications-2017/mil_marc_2017.pdf
    - The most impotant
        - https://link.springer.com/content/pdf/10.1007%2F978-3-319-59072-1.pdf (somol, pevny)
        - https://dl.acm.org/doi/pdf/10.1145/2996758.2996761
        - Stiborek - https://reader.elsevier.com/reader/sd/pii/S0957417417307170?token=492DD375F9C6F2836E0F6B394FAA2BC3E6663F3EFD26E24D7295DBFDCB870D23C3B120BCCC9731B0E6399BEFEE29C4A2


\section{HMILL framework}
- Framework definition
    - start from Pevny and Somol
    - go to Mandlik
- prior
    - Pevny, somol, Mandlik
        - https://link.springer.com/content/pdf/10.1007%2F978-3-319-59072-1.pdf (somol, pevny)
        - https://dl.acm.org/doi/pdf/10.1145/2996758.2996761
    - https://arxiv.org/pdf/2002.04059.pdf - Pevny, Dedic
    - https://arxiv.org/pdf/1906.00764.pdf - Pevny Kovarik, 
    - https://papers.nips.cc/paper/2017/file/f22e4747da1aa27e363d86d40ff442fe-Paper.pdf - same idea, but different name (just marginally)


Regardless of quite comprehensive theory which we are trying to cover in this part of the thesis hmill framework is still not all what we used (not made up of course). Our goal is quite cross-cutting and the model we are going to train we want also explain. What we mean by explanation we describe in next chapter.




Previous connection
- We focus on structured data and our tool is hmill, let's describe it
Way through this chapter
- multiple instance learning
- hierarchical mill
- prior
    - Prior in various fields
    - Prior in cybersec
Next connection
- Not necessary, but can connect to next chapter slightly (we want to explain this kind of model)

Stand on Madlik chapter, his citations (be aware), somol and Pevny


- GOALS 
- Learn the hierarchical multiple instance learning framework (HMill)
- describe theory, connect it to classical approach
- Usual usage of this type of learning - classification, regression...
- Our usecase and experiments with this kind of learning in malware classification field (prior)