\chapter{Signatures description} \label{app:signatures}
The split in the table \ref{tab:signatures_detail} denotes two groups of signatures:
\begin{itemize}
  \item Directly in \emph{report.json} in parts \emph{summary, enhanced}
  \item Not directly in \emph{report.json} in parts \emph{summary, enhanced} (some clues might be presented, but the direct information is not there)
\end{itemize}
Signature entry in original report can be seen in \ref{lst:signature_example}. Implementations of signatures could be seen on \url{https://github.com/kevoreilly/community} and example is in \ref{lst:signatureimpl}.

\begin{table}[h]
  \centering
  \caption{Used signatures and their details}
  \begin{minipage}{\linewidth}
  \begin{tabular}{p{3.5cm}p{6cm}p{3cm}p{1.3cm}}
    \toprule
    \textbf{Signature} &
    \textbf{Description} &
    \textbf{Subject} &
    \textbf{P~[\%]\footnote[1]{positive examples ratio in our dataset ($80000$ samples)}}
    \\
    \midrule
    antidebug setunhandledexceptionfilter & filters api call \emph{SetUnhandledExceptionFilter}, which enables an application to supersede the top-level exception handler of each thread of a process  (source \url{https://docs.microsoft.com/})  & API CALLS & $45$ \\
    \midrule
    copiesself & detects that currently analysed file copies itself & Dropped files & $18$ \\
    \midrule
    deletesself & detects that currently analysed file deletes/move itself or directory where placed, examining parameters of calls & API CALLS, basic file attributes (path\dots) & $27$ \\
    \midrule
    enumerates running processes & detects more than five process detail listing, saves \emph{pids} in data part & API CALLS & $16$ \\
    \midrule
    stealthtimeout & detects sequence of api calls which seems like expiration check and soon exit & API CALLS & $21$ \\
    \midrule
    useswindowsutilities & detect usage of usual windows utilities (\emph{attrib, copy, dir, echo, erase\dots}) & COMMANDS & $18$ \\
    \midrule
    removeszoneidads & detects attempts to remove evidence of file downloaded from internet by examining parameters of api calls& API CALLS & $28$ \\
    \midrule[0.3pt]
    \midrule[0.3pt]
    antisandboxsleep & detects attempts to delay the analysis task, saves \emph{pids} and time to data part & TIME, API CALLS & $39$ \\
    \midrule
    dropper & detect dropping of binary and its execution & PROCESSES, DROPPED FILES & $15$ \\
    \midrule
    invalid authenticode signature & detects authenticode signature is invalid & STATIC, DROPPED FILES & $36$ \\
    \midrule
    packerentropy & detects encrypted or compressed data using entrophy calculation & STATIC & $22$ \\
    \midrule
    stealthnetwork & detects network activity which is not detected in api calls & NETWORK & $66$ \\
    \bottomrule
  \end{tabular}
  \end{minipage}
  \label{tab:signatures_detail}
\end{table}

\begin{lstlisting}[language=json, caption={Example of \emph{signature} part entries in \emph{report.json}},captionpos=b, label={lst:signature_example}]
  {
    "name": "dead_connect",
    "description": "Attempts to connect to a dead IP:Port (1 unique times)",
    "severity": 1,
    "weight": 0,
    "confidence": 100,
    "references": [],
    "data": [
      {
        "IP": "23.238.43.43:80"
      }
    ],
    "new_data": [],
    "alert": false,
    "families": []
  },
\end{lstlisting}


\begin{lstlisting}[language=mypython, caption={Example of signature implementation},captionpos=b, label={lst:signatureimpl}]
  
  class antidebug_setunhandledexceptionfilter(Signature):
  name = "antidebug_setunhandledexceptionfilter"
  description = "SetUnhandledExceptionFilter detected (possible anti-debug)"
  severity = 1
  categories = ["anti-debug"]
  authors = ["redsand"]
  minimum = "1.3"
  evented = True

  def __init__(self, *args, **kwargs):
      Signature.__init__(self, *args, **kwargs)

  filter_apinames = set(["SetUnhandledExceptionFilter"])

  def on_call(self, call, process):
      if call["api"] == "SetUnhandledExceptionFilter":
         return True
  
\end{lstlisting}

% \begin{figure}
%     \centering
%     \begin{subfigure}{.33\textwidth}
%       \centering
%       \begin{tikzpicture}
%         \pie[text=inside, radius=2]{50/Yes, 50/No}
%     \end{tikzpicture}
%       \caption{1a}
%       \label{fig:sfig1}
%     \end{subfigure}%
%     \begin{subfigure}{.33\textwidth}
%       \centering
%       \begin{tikzpicture}
%         \pie[text=inside, radius=2]{70/Yes, 30/No}
%     \end{tikzpicture}
%       \caption{1b}
%       \label{fig:sfig2}
%     \end{subfigure}
%     \begin{subfigure}{.33\textwidth}
%         \centering
%         \begin{tikzpicture}
%             \pie[text=inside, radius=2]{20/Yes, 80/No}
%         \end{tikzpicture}
%         \caption{1c}
%         \label{fig:sfig3}
%     \end{subfigure}
%     \caption{plots of....}
%     \label{fig:fig}
% \end{figure}

% For each signature description of what is doing in more detail, what is in data part, plus histogram (in percentage)

%---------------------------------------
% More details about each signature is in appendix \todo{add appendix where each signature will by summarized, its implementation and its data part...}

% Example of its instance in signature part and description of its implementation
