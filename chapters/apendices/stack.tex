\chapter{Technologies} \label{app:technologies}
Our technology stack is quite broad, in lists below we divide technologies into two groups based on what we did use it for. We do not mention standard linux tools like \emph{shell scripts}, \emph{ssh} etc. The main part of our experimenting we spent working on Ubuntu 20.04.

In the process we usualy worked with a lot of tools we mention only those which had significant impact in our work.


\section*{Sandboxing and infrastructure}
\subsection*{Capev2 with community signatures}
A malware sandboxing tool which provides us environment to retrieve malware analysis which we used for further modelling. 

There are several relevant sources here are the most important:
\begin{itemize}
  \itemsep0em 
  \item Public instance - https://capesandbox.com/
  \item Opensource project -  https://github.com/kevoreilly/CAPEv2 (under GNU General Public License v3.0)
  \item Community extensions (e.g. \ signature implementation)
  \item Documentation - https://capev2.readthedocs.io/en/latest/
\end{itemize}

\subsection*{Virtualization}
Virtualization of sandbox machines, router machine and other related stuff was ensured by following tools:
\begin{itemize}
  \itemsep0em 
  \item \emph{Kernel-based Virtual Machine} - https://www.linux-kvm.org
  \item \emph{VirtualBox} - https://www.virtualbox.org/
  \item \emph{Windows 7} - operating system running on sandbox virtual machines
  \item \emph{Ubuntu server} - operating system running on VPN lab edge router
  \item Other tools and sources - https://github.com/doomedraven
\end{itemize}

\subsection*{Networking}
\begin{itemize}
  \itemsep0em 
  \item Ansible - https://www.ansible.com/
  \item OpenVPN - https://openvpn.net/
  \item brctl - https://linux.die.net/man/8/brctl
  \item rsyslog - https://www.rsyslog.com/
  \item fail2ban - https://www.fail2ban.org/
  \item aide - https://aide.github.io/
  \item ufw - https://help.ubuntu.com/community/UFW
\end{itemize}

\subsection*{Programming}
For programming tasks in infrastructure part we used \emph{Python 3} (see in \ref{app:attach}).
\subsection*{Others}
We are really pleased that we could use \emph{pafish} (https://github.com/a0rtega/pafish) as a testing malware sample, we used it many times\dots

\section*{Data and machine learning}
\subsection*{Julia}
For programming tasks in this part we used \emph{Julia} language - https://julialang.org/. Julia has many advantages regarding maily performance comparing to Python which could be considered as alternative. But we do not aspire to advocate this language, programming environment was mainly determined by the \emph{hmill} framework which is quite young and its first implementation is in this language by \cite{Mandlik2020}. List of the most important libraries used and their versions follows:
\begin{itemize}
  \itemsep0em 
  \item \emph{JsonGrinder.jl} - \cite{Pevny2019} (v2.1.4)
  \item \emph{Mill.jl} - \cite{Pevny2018} (v2.4.1)
  \item \emph{Flux.jl} - \cite{Innes2018a, Innes2018} (v0.11.6)
  \item \emph{EvalMetrics.jl} - https://github.com/VaclavMacha/EvalMetrics.jL (v0.2.1)
\end{itemize}
\subsection*{Computing grid system}
For resource-demanding computation we used CESNET metacenter (mentioned in acknowledgements) - https://www.metacentrum.cz/en/Sluzby/Grid/index.html.


  %-----------------------------------------
% Technology - Julia (I think I should add this in some appendix, summarize it and reference all sources used)
%   - describe why and basic features and advantages and cons
%   - Also all different codes, documented, refactored list even important libraries used in Julia

% JsonGrinder (even a little bit describe function)
% \emph{EvalMetrics.jl}

% Divide it in parts according to what we did with the infrastructure

  % Used technologies (in appendix)
%   - previous
%   - ExplainMill
%   (reference all the repositories - even cuckoo comminity)

% \todo{among appendices technologie add even (Technical background, used metascenter archicture...) and even the script
% - used technologies - add to appendix as in previous chapter
%   - \todo{reference JsonGrinder, Mill, EvalMetrics...}
%   - reference, describe functioning
% In each case, what it is, link, what we used it for