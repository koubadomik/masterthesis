\chapter{Introduction} \label{chap:intro}
\section*{Motivation}
Nowadays, we face an intense explosion of machine learning applications in various branches of human efforts. We can see applications in biology, chemistry, physics, and others. These technologies are influencing our daily life. They make it easier, faster and more enjoyable. On the other hand, we can see cases where algorithms (especially in machine learning) can control ourselves, our decisions, reasoning, and life.

If we use these computer science tools in a good way, we are often able to create something that may serve our protection. As an example, we can see the detection of threats and frauds in cybersecurity. Research and applications in this particular field are fascinating for multiple reasons, and one of them is motivation. We have to know which side we are standing on and what are the interests of our client.  In the case of fraud detection, we know that the investment is profitable only if the fraud has a significant financial impact. Not all frauds are interesting from the financial point of view because solving them also costs a lot of money. However, from an ethical point of view, every fraud should be punished. 

Similarly, network security, single device security and access control are considered to be applied later or never. Small businesses which are aiming at a specific market are not interested in some stuff which costs much money, and its main impact is preventive. The main role plays financial profit and costs reduction. Nevertheless, nobody wants privacy or data loss. It is often impossible to achieve both. We have to start with an analysis of costs, benefits, risks, their probability and impact (potential damage). This is not so evident in a technical branch as cybersecurity is.

The inseparable character of this play is malware. Let us motivate this thesis by listing several examples.

Firstly, one of the most prevalent malicious software is \emph{ransomware}. Its overall damage is estimated to be $\$20$ billion, and it increases every year \cite{purplesec2021}. The social impact is even more significant than the amount of money. We can see even attacks aiming at healthcare organizations. The first death reported following a ransomware attack was reported in 2020 \cite{Cimpanu2020}.

In Sonic Wall's 2019 report, we can find that IoT malware is becoming more common. It is caused by insufficient protection of these small devices, where we cannot provide complete malware protection. However, $127$ new IoT devices are connected to the internet every second. This leads us to estimate that by the end of 2021 we have 35 billion IoT devices connected to the internet \cite{TheIoTRu52:online}. This risk can not be mitigated easily, and malware elimination will play a significant role as it did so far.

Another convenient trend for malware is widespread encryption which becomes a standard in web traffic. Its main goal is the security of information. However, the creators of malware have much to hide and secure against the protectors, too. The encryption might inform us that the source has something that nobody else should see. A long-lasting trend of such behaviour might be suspicious. We can check if there is a justified reason, or we can at least make some conclusions. But not in a world where everything is encrypted.

$94$~\% of malware was delivered by email in 2020 \cite{Topcyber13:online}. The importance of phishing emails with malicious attachment and other social engineering techniques grows. It is cheaper to produce one sophisticated, convincing email to retrieve some information than attempt to attack a highly protected network perimeter. It also might be used to distribute malware or other threats.

In 2020 AV-atlas \cite{AVATLASM39:online} recorded over 750 million malware samples. At the end of April 2021, it was already 820 million malware samples. The majority of them are executable files attacking windows devices.

Malware research continues, and it undoubtedly will unless we can introduce a solution that is sufficiently universal and flexible to be able to detect zero-day threats (unseen). We might find a solution among machine learning models, which are often involved. However, its challenge is interpretability and explainability, not only in cybersecurity. We face the problem that the performance of the model is often significant, but we are not sure why. It is risky to deploy such a model to a situation where it can meet unseen data. High-quality security engineers do not have to be high-quality machine learning engineers. If we want to involve machine learning methods more and more, we need to be able to interpret and explain its predictions. Then we can combine knowledge from cybersecurity with the results of models and gain a better understanding. 


\section*{Goal}
The main objective of this thesis is to design a pipeline that has a malware file dataset as input and a machine learning model and its explanation as output. We want to go through the whole process, document each step and report results. Our acquisition is the process itself, it is described in detail for the reader to be able to identify problems we experienced and to be able to replicate or extend our setup.

From the assignment of this thesis, we can extract the following steps:
\begin{enumerate}
    \itemsep0em 
    \item Run several instances of CapeV2 \cite{Cape} sandbox and solve their orchestration for this experiment
    \item Capture behaviour of selected malware samples and store results
    \item Learn the hierarchical multiple instance learning framework
    \item Analyze captured data. Report basic statistics and choose appropriate features and hidden states for further modelling
    \item Using HMill, create models and identify the artefacts corresponding to different malware behaviour, report results
    \item Investigate which parts of the CapeV2 log are important to different malware behaviour
\end{enumerate}

The first step implies that we will use dynamic malware analysis to retrieve input for our machine learning model. This is motivated by a couple of thousands of sandbox \emph{JSON} reports we downloaded from the internet and examined. This dataset was not sufficient to observe significant performance. However, it was sufficient to realize that this use case is interesting for further research.

The initial task is data collection. We are about to use \url{https://bazaar.abuse.ch/} as a public data source of malware samples. We chose it because of its free access with no claims in usage and a reasonable amount of samples. We aim at \emph{Portable Executable} (PE), which does not require any additional software running on the target machine. The sandbox we want to use is \emph{CAPEv2} \cite{Cape} because first reports were also produced by this tool, and they are sufficient for analysis purposes.  It is a fork of popular \emph{Cuckoo} sandbox which is no longer maintained. We need to deal with distributed run of the sandbox to be able to collect a sufficient number of samples.

The model we want to use is \emph{hierarchical multiple instance learning} model. In \cite{Mandlik2020} authors showed that this model has significant performance modelling \emph{JSON} documents. We describe it and its alternatives in the first part of the thesis. 

After further research, we decided to model the dependence of malware \emph{signatures} on behavioural features, both presented in \emph{JSON} report. Signatures are the essential input for the original classification techniques used by the sandbox, and we want to see how well the model predicts them based on malware behaviour. We can study the implementation of signatures and their true cause, it might help us with results evaluation.

Finally, we will attempt to explain predictions by choosing a minimal subset of features that contributes to the model predictions the most. The explanation will be performed using the existing \emph{HMill} explainer. Its results will be compared to the original cause of signatures.

\section*{Thesis structure}
The thesis is divided into two main parts. In the first part, we focus on the theoretical background of our method. In the second part, we present a specific setup, our results and their discussion. At the beginning of each chapter is mentioned its particular goal. More complex structures (images, tables) are part of appendices. The thesis also has some attachments, including code, results and examples. These are described in \ref{app:attach}.

The theoretical part starts with the malware analysis theory in chapter \ref{chap:analysis} where we break down the malware itself, types of its analyses and its output. We continue in chapter \ref{chap:classification} where we describe machine learning formalism, cybersecurity context and structured data (\emph{JSON} document) usage in machine learning. Finally, chapters \ref{chap:hmill} and \ref{chap:expth} describe particular methods used in our modelling and explaining experiments.

The second part consists of two chapters. Chapter \ref{chap:infrastructure} includes description of used infrastructure and data collection process. Chapter \ref{chap:results} presents model and explainer setup, results and their discussion.




% The second part breaks down the use case we have. The first chapter \ref{chap:infrastructure} describes malware analysis sample collection, infrastructure and tools used to do that. Following two chapters contain description of the data and reasoning on how to use it and description of training process - hyperparameters, different models, conditions. Finally, last chapter is \ref{chap:expex} where we summarize the model explanation and its results. Crux of the thesis is the dicussion in the chapters \ref{chap:models} and \ref{chap:expex}where we address the model performance and explanation achievements and problems. We follow this discussion in the conlusion and in the formulation of goals of future work.


% The subject of this thesis is malware classification which is part of threat/intrusion detection. We want to use specific machine learning tool - \emph{hierarchical multiple instance learning} to train classifier on the data which we collect ourselves using dynamic malware analysis (\emph{sandboxing}). Our goal is to use data in \emph{JSON} format because this format is quite common output of reporting modules of sandboxes and usualy include majority of the information in structured form. The other reason is that the framework we are about to use shows significant results using such data.

% After we train the classifier (or several) we measure its performance and explain its predicitons. By explanation of predicition we mean using statistical method to extract the most important parts of input vector which contributed the most compared to other parts. Finally we discuss the performance of classifiers and the output of explanation.

% The model could be used to detect malicious code execution based on its behavior and later the explanation can be used to generalize some of our deterministic ways to detect malware (family assignment, signature marking) or just to explain what is for some specific malware family some generic group of malicious programs.

% There are experiments in each part of our task but we would like to examine the whole pipeline to see if we are able to go from plain malware sample to predictor which is explainable. We do not aspire to dig deep in some specific part rather we want to connect each step and identify problems which are in the whole process. We want to see it from a greater prespective. The whole process itself is acquisition and we want to demostrate all the steps.
% \todo{We can add some references}
% \section{Goals}




%%---------------------------------------------------------
% The structure of this thesis follows the steps which we defined in assignement but we solve theoretical point of view and our use case into two parts of this thesis.
% Text is organized such that each chapter focuses on specific goal from the assignement. Firstly we describe theory and prior work, further our conditions, experiments and conclusions.

% The main goal of this thesis which we are aiming at is to capture and analyze artifacts of dynamic malware analyses and draw conclusions in each part of the process. 

% - motivation
% - Motivation behind the modelling itself (classifying itself)
%   - classifying accordig to dynamic analyses, prior arts....
%   - My thesis is more practical, so the main goal is not to compare my results to another ones, but just demonstrate acuuracy of such classifier and everything around like for instance sandboxing, exaplaining
% - goals
% - structure of the thesis
