https://www.sciencedirect.com/science/article/pii/S1084804519303868 - good overview and taxonomy of malware

\todo{describe part one goal - We involve more theory and prior work research, in following chapters we discuss our case.}
In the first part of this thesis we focus the comprehensive theory which is needed to absorb in order to interpret results. Not everything is disassembled in detail we go only there where we find something useful for our thesis. This is really based on experience so if the reader is interested only in results used formalism and notation in theory part are important only.

\chapter{Malware analysis} \label{chap:analysis}
The data which we are performing analysis on are malware analysis results. Specifically those are dynamic analysis of running Portable executable (PE) files in sandbox environment. In this chapter we cover theory of malware analysis and everything related. 

\todo{add some motivation example or latest results... - known malware attacks..}

Philosophy could be walking around risk management, probability of risk and resulting damage... Maybe more examples of situations where we know how much it cost (Cambridge analytica, ransomware...)
Maybe something about truth and its protection in the world of lies and fake news
Confidentiality, integrity and availability

Use found books or post-hoc go through some articles and formulate the motivation part...

\section{JSON notation} \label{sec:json_notation}
(define json notation in short paragraph)

\section{Challenges}
Data quality, bias, antisandbox detection... where to find samples? (we should definitely mention motivation why we are collecting our own data), public data (this is quite risky to publish in general), everything is fast (zero days), Are those anomalies or malicious?, Imbalanced data sets, encryption everywhere - what to trust..., cryptocurrencies, iot, cloud

https://www.groundai.com/project/machine-learning-in-cyber-security-problems-challenges-and-data-sets/1
https://www.readitquik.com/articles/security-2/cybersecurity-challenges-that-need-to-be-on-your-radar-right-now/


\section{Malware}
Generally types of attacks and just the intro, to settle the malware into wider context
Definition of cleanware and malware
Types, Families
https://en.wikipedia.org/wiki/Malware
Find in books!

\section{Portable executable files}
Malware is defined widely and is not limited to specific kind of file format. In our research we decided to investigate only Portable Executable (PE) files. It is because we want to create dataset of malware running on windows machines with no extra extension. 

In general we can find various filetypes. In list below we list some examples with basic principle how particular type is used for system intrusion or harm. From the definition of malicious code we can derive that whichever filetype could by used but some are more usual than another (easier to use).
\begin{enumerate}
  \item Portable executable
  \item Portable Document Format
  \item Microsoft office formats
  \item Compressed
\end{enumerate}

https://en.wikipedia.org/wiki/Portable_Executable
https://docs.microsoft.com/en-us/windows/win32/debug/pe-format
https://github.com/corkami/pics/blob/master/binary/pe101/README.md

\section{Malware analysis}
introduction

https://en.wikipedia.org/wiki/Malware_analysis
Find in books!

\subsection{Dynamic}
https://publications.sba-research.org/publications/malware_survey.pdf
https://link.springer.com/article/10.1007/s11416-007-0074-9

mention especially parts which gives more information about the run (mutexes, files, api calls,.. - to be able to reference from data chapter)

- Summary of potential sorces of metadata
○ Abuse.ch - https://bazaar.abuse.ch/api/#api_key
○ VirusTotal - https://developers.virustotal.com/v3.0/reference#file-info
○ Metadefender - https://onlinehelp.opswat.com/mdcloud/3.1_Retrieving_scan_reports_using_a_data_hash.html
- Interesting
○ https://app.any.run/
○ https://virusscan.jotti.org/
○ https://virscan.org/
○ https://mzrst.com/
○ https://github.com/maliceio/malice
https://www.hybrid-analysis.com/

\subsection{Static}
○ Retdec - static analyser
§ Not so well documented..
§ I am studying the output, there is decompiled code and there some metada
§ I will use as it is out of box - decompiled code and some metadata
○ Manalyze
§ Another metadata like clamav signatures and something else


\section{Sandboxing}
existing solutions and results...
find basics in books

api hooking, dumping import reconstruction debugging static parsing - https://www.youtube.com/watch?v=qEwBGGgWgOM

to appendix we can add also web interface of cuckoo


\section{Produced data}
mention signatures
usual format which are produced by sandbox and what we can find there
Reference even docs - no problem, not only books and papers
Conclusion should be that those data are complicated and their structure also keep some information (order, structure...), that is why we want to focus on structured data and especially json.


\section{Prior}
works about malware analysis
works that collected data for machine learning purposes
Mandlik, Stiborek, everybody who used cuckoo or other sandbox data (reference for what and their results)

file:///C:/Users/domia/Downloads/Imad-Saiida-IJCNIS-V11-N6-1.pdf
https://web.archive.org/web/20160418151823/http://www.ijarcsse.com/docs/papers/Volume_3/4_April2013/V3I4-0371.pdf
https://reader.elsevier.com/reader/sd/pii/S1383762120301442?token=81F1AB06FF0FE40D1B7745234269280FCF2CB0979140E84DB35D6E04A2C622FE6EDA2DE4DA0630E1D73044E3D84A722F&originRegion=eu-west-1&originCreation=20210401090943
https://iopscience.iop.org/article/10.1088/1742-6596/1140/1/012042/pdf
file:///C:/Users/domia/Downloads/JCIT4024PPL.pdf
https://dl.acm.org/doi/abs/10.1145/2046614.2046618?casa_token=MXpfHiylCZcAAAAA:szRMPWfKXTl4wxP2-h32eknCg5dzM2t7RxGjywiJDksmT5FqcUY7pPrIBZchv26HUe3Lwubu5Hru

file:///C:/Users/domia/Downloads/088851962.pdf


%%------------------------------------------

Previous connection
- Nothing before
Way through this chapter
- Malware definition and types
- Malware analysis
- Challenges
- Data - input, output...
- prior
Next connection
- We are interested in modeling the data using modern approaches, we know that the data are often structured somehow and that is what we will solve

- GOALS
  - Run several instances of CapeV2 sandbox and solve their orchestration for this experiment
  - Capture behavior of selected malware samples in CapeV2 sandbox and store results
- Theory part - types, conditions, bias, Malware types, signatures....
- Sandboxing
- Previous experiments
- Our goal in this part
- Data collection for ML purposes in general