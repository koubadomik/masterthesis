\chapter{Conclusions} \label{chap:concl}
The main objective of this thesis was to design a pipeline that has a malware dataset as input and a machine learning model and its explanation as output. The whole process was motivated by high accuracy model interpretability to achieve greater compliance of machine learning and cybersecurity.

Theory background and methods are summarized in the first part of the thesis. The setup, experienced problems, results, and their discussion are in the second part.

We set up eight physical machines with the \emph{CAPEv2} sandbox in two different setups --- with internet and without internet connection. Using the open-source sandbox and our programs, we collected dynamic malware analyses for 80~000 malware samples retrieved from MalwareBazaar\footnote{https://bazaar.abuse.ch/}. We reported the problems experienced during the data collection process and the description of the whole setup, including our code.

We used \emph{JSON} reports of sandbox as an input for \emph{Hierarchical multiple instance} framework \cite{Mandlik2020}, the choice of this technology can be justified by its ability to model \emph{JSON} documents. The classification model features are behavioural parts, and predicted classes are malware signatures, both included in the original \emph{JSON} report. 
To evaluate our models better, we investigated the original signature's implementation and found out their true cause. We created a binary classifier for each of the chosen signatures (overall 12). We observed how each model performs in the context of the true cause, which was or was not among the model's features. Nine classifiers had a balanced accuracy of more than 90\%. We reported and discussed individual results.

Finally, we experimented with the model explaining. Even though there might be hundreds of entries from the original behavioural reports used as a feature set, the explainer only provides 3--5 entries as an explanation for each of the nine explained models. It is evident from our observations that some models are strongly involved with the original signature's cause. There are cases where the model used different behavioural features with high accuracy. We reported and discussed all results.

Despite the significant amount of work we faced during the sandboxing, we managed to meet the goals of this thesis. We wanted our experiments to be repeatable and therefore the source code and other technicalities are in the attachment of this thesis.

Regarding the implementation, both the realization of the whole pipeline as well as its explanation is considered successful, we further discussed and reported all of the results. In addition, we mention individual/specific issues faced during the work along with ideas for future work.

\section*{Future work}

