\chapter{Malware classification}
We can expose a lot of information from malware analysis result. It contains api calls, network requests, downloaded payload and much more. This could be used to explain directly what is going on during the run. Sandbox itself is often labelling malware samples using so called signatures. These signatures are deterministically assigned and they usually sign that particular malware sample is doing something suspicious (we will describe later). Based on signatures we can later distinguish between different kinds of malware or between malware and cleanware - classify. Besides this deterministic way modern approaches from machine learning are often used.

In this chapter we focus on classification problem in general and later we describe malware classification. We involve more theory and prior work research, in following chapters we discuss our case.

In previous chapter we described different data outputs which are often produces by different kinds of analysis. That why later in this chapter we focus on models which are designed to for such data.

\section{Classification in general}
\todo{Add some examples of use of ML and AI in cyber security field}


\todo{I think we want to aim only on stastistical machine learning, so let's define it at the beginning}
The largest challenge in machine learning theory is very often used formalism and framework and point of view which we consider. So firstly we would like to summarize where in this tangle we are and where we are going.

As stated by \citet{GoodBengCour16} the usual aproach in distinguishing tasks for machine learning algorithms are most often divided in categories based on expected sample processing result. By sample we mean collection of features which tends to be represented as $$TODO: add vector on R or tensor on R$$. As an example we can introduce images with fixed dimension but in our usecase even serialized json files or pcap files.

In this section we are aiming at standard machine learning approach and tasks formulation, later we will define task which is actual for our work.

\todo{add scheme of machine learning algorithm in classical manner, so features to hidden states (sometimes label) mapping, define some formalism for math signing}

\paragraph{Usual and most seen categories are:}
\begin{enumerate}
    \item Classfication \todo{describe, add some math}
    \item Regression \todo{describe, add some math}
\end{enumerate}

There are other categories like transcription \cite{GoodBengCour16} or anomaly detection \cite{Chandola2009} which is quite often seen in cyber security field \todo{add example of this kind of task in cyber security https://link.springer.com/article/10.1007/s10994-014-5473-9}. And many other defined problems mentioned in \cite{GoodBengCour16} or \cite{zhang2020dive}.

Based on presence of hidden states and other conditions we can further distinguish between supervised and unsupervised learning tasks. \cite{zhang2020dive}. One of the biggest challenges of machine learning nowadays are labelled data, where every sample consists of features and appropriate hidden state - dataset for supervised learning. That is one reason why we want to be able to face even situations where the hidden states are not presented - dataset for unsupervised learning. \todo{add some paper where they talk about challenges like this} Those situations we can very often see in cyber security field because in principle attacker or any kind of distractor often wants to hide acts such that protector is not able to identify. So the hidden state itself could be assigned rather based on further data collection or never.\todo{find something about unsupervised learning in cyber security as an example https://technative.io/why-unsupervised-machine-learning-is-the-future-of-cybersecurity/, https://ieeexplore.ieee.org/abstract/document/8268746?casa_token=ctNKO2pD640AAAAA:XoFuD8SHvbzMszA1nHciHIiaetBCG7fN47zUNZDMYv2D5oZUqhGuq9FF_ETI9Rhf7QtsTYagkw, https://arxiv.org/abs/1812.07858}.
There are also cases where we talk about semi-supervised learning \todo{define it and reference some source for example GoodBeng is mentioning it}.

\todo{from this we can go to missing values in data and maybe even to various dimesion data learning}

The oposite point of view on non-ideal datasets are those where features are missing. \todo{http://www.iiis.org/CDs2008/CD2008SCI/SCI2008/PapersPdf/S507DT.pdf is source of information here}

Later we will mention other examples of tasks mainly different in the assumtions for input data.

\todo{Here we can find something else https://machinelearningmastery.com/a-tour-of-machine-learning-algorithms/}

Our task is to classify samples so in further writing as model we reference classification model.

train, test, validation and hyperparameters, overfitting, underfitting (https://www.deeplearningbook.org/contents/ml.html)


\section{Classification models}
There are various approaches to classification. In this section we describe some of basic models and their usual output and its intepretation. 

reference discriminative models because this paper is talking about it also - https://dl.acm.org/doi/pdf/10.1145/2996758.2996761 and we will work with that in further sections

Mention on what kind of optimization it stands
https://analyticsindiamag.com/7-types-classification-algorithms/
https://www.edureka.co/blog/classification-algorithms/#typesofclassificationalgorithms
https://dzone.com/articles/introduction-to-classification-algorithms

\section{Model Evaluation}
Based on what we are solving and what we want to comapare to we have to choose proper metrics to evaluate our classifier. Majority of those techniques does not depend on type of classifier we have. The classifier itself is just blackbox which we are evaluating based on simply classificcation results.
Mention training and learning case and metrics in different context
general measures
nice-to-have: early stopping based on metrics like those
https://www.researchgate.net/profile/Mohammad-Hossin/publication/275224157_A_Review_on_Evaluation_Metrics_for_Data_Classification_Evaluations/links/57b2c95008ae95f9d8f6154f/A-Review-on-Evaluation-Metrics-for-Data-Classification-Evaluations.pdf

measures important in cyber security
https://scholarcommons.usf.edu/cgi/viewcontent.cgi?article=1005&context=mca
https://ntnuopen.ntnu.no/ntnu-xmlui/bitstream/handle/11250/2730527/Hameed%252C%2BIbr.pdf?sequence=2&isAllowed=y
https://owasp.org/www-community/controls/Intrusion_Detection
On the other hand I will write even my own reasoning in particular in domain of cyber security what does it mean to misclassify as positive or negative example...
In majority of papers like this https://core.ac.uk/download/pdf/77064684.pdf, they just state what they use fo accuracy and done. Or here :D https://core.ac.uk/download/pdf/80994982.pdf
this is aiming at false positives in cyber security https://ieeexplore.ieee.org/document/7560382
Goal obviously is to keep high false negative and lowen false positive beacuse both metrics are really important in network security
https://www.ccdcoe.org/uploads/2018/10/Art-19-On-the-Effectiveness-of-Machine-and-Deep-Learning-for-Cyber-Security.pdf - section fn and fp



\section{Data characteristics}
Model which we choose for given task is based on several facts like domain which we are dealing with and mainly on the data we have.
Input to usual model is binary vector or rather tensor, but those tensor could be encoding of different things
Basic point of view is to distinguish between two basic types of variables - continous and discrete (https://machinelearningmastery.com/feature-selection-with-real-and-categorical-data/)
Highlevel view will refer to images, time series, graphs (json...), just binary...
Maybe we can talk about fixed dimension, various dimension

Write more about json files and generally describe graph inputs

\section{Neural Nets}
In the section models \todo{reference} we briefly described different kinds of models. One specific model we would like to discuss more copmrehensively - neural nets. The reason is that our method is building on top of this approach.

Minibatches (random subsampling...), hyper parameters

Neural nets are machine learning phenomenon mainly because its robustness and not so demanding data assumptions. \todo{examples of use and some reference to this statement, https://stats.stackexchange.com/questions/467103/do-neural-networks-make-assumptions-about-data-and-when-to-use-standardization}.
general even math!
https://www.deeplearningbook.org/contents/ml.html
http://neuralnetworksanddeeplearning.com/chap1.html
http://playground.tensorflow.org/#activation=relu&batchSize=30&dataset=spiral&regDataset=reg-plane&learningRate=0.3&regularizationRate=0&noise=0&networkShape=4,2&seed=0.87951&showTestData=false&discretize=false&percTrainData=50&x=true&y=true&xTimesY=false&xSquared=false&ySquared=false&cosX=false&sinX=false&cosY=false&sinY=false&collectStats=false&problem=classification&initZero=false&hideText=false

optimization theory, gradient descent, stochastic gradient descent, backprop
cs.huji.ac.il/~shais/UnderstandingMachineLearning/understanding-machine-learning-theory-algorithms.pdf
https://stats.stackexchange.com/questions/186091/what-loss-function-should-i-use-for-binary-detection-in-face-non-face-detection

Loss functions and nets used for classification
https://machinelearningmastery.com/how-to-choose-loss-functions-when-training-deep-learning-neural-networks/

\section{Classifying based on graph structured data}
Real world use cases provide often more complex dastasets than just fixed dimension matrices or images. As we mentioned our data are often stored as JSON files. Those files could be formally seen as tree-structured inputs for machine learning algorithms. 
\subsection{Problem}
https://www.quora.com/What-is-structured-data-in-the-context-of-machine-learning

https://www.naftaliharris.com/blog/machine-learning-json/

\subsection{Prior}
https://arxiv.org/pdf/1506.05163.pdf
https://arxiv.org/pdf/1911.08756v3.pdf
https://www.cs.uoregon.edu/Reports/AREA-201706-Riazi.pdf
https://core.ac.uk/download/pdf/11011889.pdf
Hmill
Graphical models
Graph neural networks
    Also seen here file:///C:/Users/domia/Downloads/Explainability_ICML2021.pdf
See in Mandlik

\section{Classification in cyber security, classifying malwawre}
file:///C:/Users/domia/Downloads/CoRR2018_submission_v3.pdf
https://www.sciencedirect.com/science/article/pii/S1084804519303868
Finally we can not skip our domain and results in this field. 
Kinds of classifications in cyber security and their description
Intrusion detection, network traffic attacks and anomalies and malware X cleanware classification,...
We can classify different things like malware X cleanware, malware family (not so useful in detection but in further learning)
Problems of cyber security field
Based on Dynamic, static, other kind of input data...
SURELY MENTION RETRAIN PROBLEM AND reason that this is true but machine learning algorithm could be much more efficient (this is kind of motivation of our work)
https://www.ccdcoe.org/uploads/2018/10/Art-19-On-the-Effectiveness-of-Machine-and-Deep-Learning-for-Cyber-Security.pdf (eventually their references - some interesting research results, good for prior)
https://www.groundai.com/project/machine-learning-in-cyber-security-problems-challenges-and-data-sets/1
https://arxiv.org/pdf/1910.11376.pdf

If I finally go through this summary report I will be save with this part hopefully - https://www.jstor.org/stable/resrep22692?seq=53#metadata_info_tab_contents, this is great summary for modern approaches and papers in cybersec, I can use it even somewhere else for example in intro of this chapter and whole thesis
For example one part was focused on types of classifiers - in machine learning is quite big difference between zero-days and known malware...

Using N-grams in detection https://www.researchgate.net/publication/262366662_A_Close_Look_on_n_-Grams_in_Intrusion_Detection_Anomaly_Detection_vs_Classification

Malware classification in more detail

not so interesting can be added between other examples - https://www.researchgate.net/publication/312964059_Malware_Classification_Based_on_Dynamic_Behavior, https://link.springer.com/content/pdf/10.1631/FITEE.1601325.pdf, https://dspace.cvut.cz/bitstream/handle/10467/87850/F3-DP-2020-Dvorak-Stepan-dvorast6.pdf?sequence=-1&isAllowed=y (mention graph neural networks)

Using static analysis https://dl.acm.org/doi/pdf/10.1145/2402599.2402604?casa_token=Kv3xJb3iPssAAAAA:rm6bhWnusg7eEvalXNveaJXILphAxhpZHGS6OxpDk36na4q5u9RpZ3gM83IMQWq1QQ0aEIVoeyZN

family classification - https://arxiv.org/pdf/1912.11249v1.pdf

Recurrent networks - https://ieeexplore.ieee.org/stamp/stamp.jsp?tp=&arnumber=7178304

HMILL in cyber security - https://arxiv.org/pdf/2002.04059.pdf, mandlik, Pevny

\subsection{Problem}
\subsection{Prior}


\todo{add conclusion of this chapter and connection to next chapter - hmill}




Previous connection
- One of conclusions of previous chapter is that that character of data is often structural as graph or tree (json, xml,...)
Way through this chapter
- Generally about machine learning challenges and types of tasks
- Classification models and their evaluation
- Neural networks
- Classifying structured data as jsons - models
    - hmill and other techniques - reference next chapter
- classification task in cyber security
- Prior on this topic - cyber security!
Next connection
- Hmill is one what we want to use to create model of the data for classification


\cite{GoodBengCour16}
Potentially:
- general points at the beggining
    - Increasing model sizes
    - Increasing Accuracy, Complexity and Real-World Impact
    - Gradient-Based Optimization
    - Stochastic Gradient Descent
    - Challenges Motivating Deep Learning
- More Sure
    - Supervised X Unsupervised
    - The Task, T - classification, regression, others exists
    - The Performance Measure, P
    - Capacity, Overfitting and Underfitting
    - Hyperparameters and Validation Sets
    - Bias and variance - Trading off Bias and Variance to Minimize Mean Squared
    Error

\cite{Bishop2006}
 - generally the introduction to classification models


- Classification problems - binary, multiclass, multilabel...
- Theory
- Single label, multilabel

- evaluation of classifiers - more or less independent on model choose
  - Confusion matrix
  - F-score, train accuracy, test accuracy, loss function, plots
        - AUC, ROC?
  - What is important metric during malware classification and why


- learning classifier from graph data - json files,...
- possible models - based mainly on the data structure we have and use case we are modelling (find how to choose proper model)
    - among them even neural networks and refer to mill/hmil in next chapter - connect to goal of this thesis, use this framework to classify malware - go on with next part
- (one of them should be using neural nets - describe deeply usage of loss function in both cases - single label, multi label...)
- hyper parameters...
- minibatch gradient descent, gradient descent itself

- Machine learning in cyber security in general
- General approaches to malware classfication - theory
    - classifying type of malware or some kind of behavior X classifying malware Vs. cleanware
    - based on dynamic/static anysis results...
    - get to neural network and finally to mill - stiborek and other applications in cyber security (Mandlik, Pevny, Dedic...) and reference next chapter.


- Prior work
Our goal is not to compare, mentions are just for reference to similar work, mention practical papers used above

nice-to have:
- (Overfitting, early stopping)

Remainders:
Malware analysis results could be used
Following data collection and processing next task is  
Based on type of analysis and characteristics of collected data}
