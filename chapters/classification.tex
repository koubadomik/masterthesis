\chapter{Malware classification}
We can expose a lot of information from malware analysis result. It contains api calls, network requests, downloaded payload and much more. This could be used to explain directly what is going on during the run. Sandbox itself is often labelling malware samples using so called signatures. These signatures are deterministically assigned amd usually sign that particular malware example is doing something suspicious (more later). Based on signatures we can later distinguish between different kinds of malware or between malware and cleanware - classify. Besides this deterministic way modern approaches from machine learning are often used.

This capture focus on classification problem in general and later in case of malware. We involved more theory and prior work research in following chapters we discuss our case.

\todo[inline]{
    Remainders:
    Malware analysis results could be used
    Following data collection and processing next task is  
    Based on type of analysis and characteristics of collected data
}
