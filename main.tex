\documentclass[12pt,a4paper,twoside]{book}
\usepackage[utf8]{inputenc}
\usepackage{setspace}
\usepackage{amsmath,amsfonts,amssymb}
\usepackage{cite}
\usepackage{algorithm,algpseudocode}
\usepackage{setspace}
\usepackage{bibentry}
\usepackage[bottom]{footmisc}
\usepackage[
a4paper,
twoside,
%bindingoffset=1cm,
inner=2cm,
outer=2cm,
top=3.5cm,
bottom=3.5cm,
headsep=1cm
]{geometry}
\usepackage{charter}
\usepackage[sort,numbers]{natbib}
%\usepackage{nyul_thesis}
\usepackage{booktabs,caption,multicol,hhline,tikz,multirow,array}
\usetikzlibrary{arrows.meta}
\usepackage{colortbl}
\usepackage{arydshln}
\usepackage{url}
\usepackage{subcaption}
\usepackage{caption}
\usepackage{graphicx, multirow, array,amsmath}
\usepackage{tabularx}
\usepackage{hyperref}
\usepackage{enumitem}
\usepackage[acronym, nopostdot]{glossaries}

\newcolumntype{C}[1]{>{\centering\arraybackslash}p{#1}}

% \newcolumntype{C}[1]{>{\centering\let\newline\\\arraybackslash\hspace{0pt}}m{#1}}
\newcolumntype{M}[1]{>{\centering\arraybackslash}m{#1}}
\newcolumntype{N}{@{}m{0pt}@{}}
\setlength{\belowcaptionskip}{10pt plus 3pt minus 2pt}
\DeclareMathOperator*{\argmin}{arg\,min}
\DeclareMathOperator*{\argmax}{arg\,max}

\usepackage{fancyhdr}
\fancypagestyle{plain}{%
	\fancyhead{}%
	\renewcommand{\headrulewidth}{0pt}%
	\renewcommand{\footrulewidth}{0pt}%
	\fancyfoot[C]{\bfseries\thepage}}
\fancypagestyle{mainmatter}{%
	\fancyhf{}
	\renewcommand{\headrulewidth}{0.4pt}%
	\renewcommand{\footrulewidth}{0pt}%
	\fancyhead[LE,RO]{\bfseries\thepage}
	\fancyhead[LO]{\bfseries\nouppercase\rightmark}
	\fancyhead[RE]{\bfseries\nouppercase\leftmark}}
\fancypagestyle{frontmatter}{%
	\fancyhead{}%
	\renewcommand{\headrulewidth}{0pt}%
	\renewcommand{\footrulewidth}{0pt}%
	\fancyfoot[C]{\bfseries\thepage}}

\captionsetup[figure]{labelfont={bf},textfont={it}}
\captionsetup[table]{labelfont={bf},textfont={it}}


%----------------------------------------------------------------------------------------
%	THESIS INFORMATION
%----------------------------------------------------------------------------------------

\makeglossaries

\begin{document}
%\UseRawInputEncoding

\setstretch{1.1}

%\nobibliography*
%\newcounter{cont}

\frontmatter
\renewcommand{\chaptermark}[1]{\markboth{#1}{}}
\renewcommand{\sectionmark}[1]{\markright{\thesection\ #1}}
\pagestyle{frontmatter}


\thispagestyle{empty}

\begin{center}
	\vfill
	\vspace*{3.5cm}
	
	\begin{spacing}{2}
		{\Huge \textbf{Cape sandbox dataset collection and analyses}}
	\end{spacing}
	
	\vspace*{2cm}
	
	{\Large Diploma Thesis}
	
%	\vspace{2.5cm}
	\vfill
	
	{\textbf{\Large Bc. Dominik Kouba}}
	
	\vspace{0.25cm}
	
	{\Large Supervisor: doc. Ing. Tomáš Pevný, Ph.D}
	
%	\vspace{2cm}
	\vfill
	
	{\Large 2020}
%	
%	\vspace{0.25cm}
%	
%	{\large Department of DEPART}
%	
%	\vspace{0.25cm}
%	
%	{\large Faculty of Science and Informatics}
%	
%	\vspace{0.25cm}
%	
%	{\large University of Szeged}
	
%	\vspace{4cm}
	\vfill
	
	\includegraphics[width=0.6\linewidth]{Figures/logo}
	\vspace{0.5cm}
	
	{\Large Department of Computer Science}
	
\end{center}
 


\include{ack}

%\chapter*{Acknowledgments}
%\markboth{}{}
\vspace*{\fill}
\section*{Author statement}
First of all, I would like to thank my supervisor, Firstname Lastname, for directing my PhD studies. I would also like to thank my colleagues and friends who helped me to realize the results presented here and to enjoy the period of my studies. 
Last, but not least, I wish to thank my wife and family for their constant love and support. 

\section*{Prohlášení autora}
First of all, I would like to thank my supervisor, Firstname Lastname, for directing my PhD studies. I would also like to thank my colleagues and friends who helped me to realize the results presented here and to enjoy the period of my studies. 
Last, but not least, I wish to thank my wife and family for their constant love and support. 



\include{abstract_cs}

\vspace*{\fill}
\section*{Abstract}
Even though machine learning algorithms already play a significant role in data science, many current
methods pose unrealistic assumptions on input data. The application of such methods is difficult due
to incompatible data formats, or heterogeneous, hierarchical or entirely missing data fragments in the
dataset. As a solution, we propose a ver
\vspace*{\fill}

\begingroup
\let\cleardoublepage\clearpage
\tableofcontents
\listoffigures
\listoftables
\endgroup




%\newpage
%\thispagestyle{empty}
%\mbox{}



% Empty pages

%\newpage
%\thispagestyle{empty}
%\mbox{}

% \newpage
% \thispagestyle{empty}
% \mbox{}

% Main


\renewcommand{\chaptermark}[1]{\markboth{#1}{}}
\renewcommand{\sectionmark}[1]{\markright{\thesection\ #1}}
\mainmatter
\pagestyle{mainmatter}



% List of Acronyms
\newacronym{gcd}{GCD}{Greatest Common Divisor}
\newacronym{lcm}{LCM}{Least Common Multiple}

% Acronym List
\printglossary[type=\acronymtype, title=Abbreviations]

% Intro:
\include{Chapters/intro_section}

\section{Examples}

% Acronym examples
Given a set of numbers, there are elementary methods to compute 
its \acrlong{gcd}, which is abbreviated \acrshort{gcd}. This process 
is similar to that used for the \acrfull{lcm}.

% Figure example:
Examples of figure:
\begin{figure}[h]
\caption{Example of figure}
\centering
\includegraphics[width=0.3\linewidth]{Figures/szte_logo.jpg}
\end{figure}

% Table example:
Examples of table:
\begin{table}[h]
\centering
\begin{tabular}{|l|l|l|}
\hline
\#    & Col 1 & Col2 \\ \hline
Row 1 & 0     & 1    \\ \hline
Row 2 & 2     & 3    \\ \hline
\end{tabular}
\caption{Example of table}
\label{main:table_1}
\end{table}

Example of equations:
% Equations example
\begin{equation}
    x_{test} = b^2 + c^2
\end{equation}

% Thesis:
\include{Chapters/Thesis1/chapter_2} 
\include{Chapters/Thesis2/chapter_3}
\include{Chapters/Thesis3/chapter_4}
% References:
\include{Chapters/references}

\backmatter

% Bibliography:
\bibliographystyle{plain}
\addcontentsline{toc}{chapter}{Bibliography}
\bibliography{refs}

% Summary
\include{Chapters/summaries}
\include{Chapters/osszefoglalok}
\include{Chapters/my_publications}

% Acknowledgement


% Empty page
\thispagestyle{empty}

\end{document}  
