\documentclass[11pt,twoside,a4paper]{book}  
\usepackage[english]{babel}
\usepackage[T1]{fontenc} 				% pouzije EC fonty
\usepackage[utf8]{inputenc} 			% utf8 kódování vstupu 
\usepackage[square, numbers]{natbib}	% sazba pouzite literatury
\usepackage{fancyhdr}					% tisk hlaviček a patiček stránek
\usepackage{nomencl} 					% umožňuje snadno definovat zkratky a jejich seznam

\usepackage{charter}					% font
\usepackage{pdfpages}					% inserting pdfs


%%%%%%%%%%%%%%%%%%%%%%%%%%%%%%%%%%%%%%%%%%%%%%%%%%%%%%%%%%%%%%%
% informace o práci
\newcommand\WorkTitle{Analyzing the execution of malware in a sandbox using hierarchical multiple instance learning}
\newcommand\FirstandFamilyName{Bc. Dominik Kouba}
\newcommand\Supervisor{doc. Ing. Tomáš Pevný, Ph.D.}
\newcommand\TypeOfWork{Master's Thesis}		
\newcommand\StudProgram{Open Informatics}
\newcommand\StudBranch{Cyber security}

%%%%%%%%%%%%%%%%%%%%%%%%%%%%%%%%%%%%%%%%%%%%%%%%%%%%%%%%%%%%%%%
% imports
\usepackage{graphicx}					% pro vkládání obrázků
\usepackage{k336_thesis_macros} 		% specialni makra pro formatovani DP a BP
\usepackage[
pdftitle={\WorkTitle},				% nastaví v informacích o pdf název
pdfauthor={\FirstandFamilyName},	% nastaví v informacích o pdf autora
colorlinks=true,					% před tiskem doporučujeme nastavit na false, aby odkazy a url nebyly šedé při ČB tisku
breaklinks=true,
urlcolor=red,
citecolor=blue,
linkcolor=blue,
unicode=true,
]
{hyperref}								% pro zobrazování "prokliknutelných" linků 

% rozšiřující importy
\usepackage{listings} 			%slouží pro tisk zdrojových kódů se syntax higlighting
\usepackage{algorithmicx} 		%slouží pro zápis algoritmů
\usepackage{algpseudocode} 		%slouží pro výpis pseudokódu
%%%%%%%%%%%%%%%%%%%%%%%%%%%%%%%%%%%%%%%%%%%%%%%%%%%%%%%%%%%%%%%
% příkazy šablony
\makenomenclature			% při překladu zajistí vytvoření pracovního souboru se seznamem zkratek
\let\oldUrl\url				% url adresy budou zobrazeny: <url> 
\renewcommand\url[1]{<\texttt{\oldUrl{#1}}>}

%%%%%%%%%%%%%%%%%%%%%%%%%%%%%%%%%%%%%%%%%%%%%%%%%%%%%%%%%%%%%%%
% vaše vlastní příkazy
\newcommand*{\nomExpl}[2]{#2 (#1)\nomenclature{#1}{#2}} 	% usnadňuje zápis zkratek : Slova ke Zkrácení (SZ)
\newcommand*{\nom}[2]{#1\nomenclature{#1}{#2}} 			% usnadňuje zápis zkratek : SZ

%%%%%%%%%%%%%%%%%%%%%%%%%%%%%%%%%%%%%%%%%%%%%%%%%%%%%%%%%%%%%%%
% vlastní dokument
%%%%%%%%%%%%%%%%%%%%%%%%%%%%%%%%%%%%%%%%%%%%%%%%%%%%%%%%%%%%%%%
\begin{document}
	\selectlanguage{english}
	\translate

	% Assignment
	{
		%\includepdf[pages=-]{pdfs/zadani.pdf}
		\newpage
	}

	% Title page 
	\coverpagestarts

	% Acknowledgements 
	\acknowledgements
	\noindent
	Zde můžete napsat své poděkování, pokud chcete a máte komu děkovat.

	% Declaration 
	\declaration{In Prague on May 15, 2021}

 	%Abstract 
	\abstractpage

	Translation of Czech abstract into English.

	\vglue60mm

	\noindent{\Huge \textbf{Abstrakt}}
	\vskip 2.75\baselineskip

	\noindent
	Abstrakt práce by měl velmi stručně vystihovat její obsah. Tedy čím se práce zabývá a co je jejím výsledkem/přínosem.

	\noindent
	Očekávají se cca 1 -- 2 odstavce, maximálně půl stránky.

	%%%%%%%%%%%%%%%%%%%%%%%%%%    
	% obsahy a seznamy
	\tableofcontents		% Obsah / Table of Contents 

	% pokud v práci nejsou obrázky nebo tabulky - odstraňte jejich seznam
	\listoffigures			% Obsah / Table of Contents 
	\listoftables			% Seznam tabulek / List of Tables

	%%%%%%%%%%%%%%%%%%%%%%%%%% 
	% TEXT
	\mainbodystarts

	%Chapters
	\chapter{Introduction} \label{chap:intro}
\section*{Motivation}
In this day and age, we face an intense explosion of machine learning applications in various branches of human efforts such as biology, chemistry, physics, and others. These technologies widely influence our daily lives and make them immensely more convenient, faster, and more enjoyable. On the other hand, there are many cases where algorithms (especially in machine learning) can control our decisions, reasoning, and life.

If we use these computer science tools appropriately, we can often create something that may serve our protection. We can take the detection of threats and frauds in cybersecurity as a perfect example, research and applications in this particular field are fascinating for multiple reasons, and one of them is our motivation. We have to know which side we are standing on and what the interests of our clients are. In the case of fraud detection, we know that the investment is profitable only if the fraud has a significant financial impact. Not all frauds are interesting from a financial point of view because solving them also costs much money. However, from an ethical point of view, every fraud should be punished. Similarly, network security, single device security, and access control are often disregarded. Small businesses targeting a specific market are not interested in costly services whose impact is mainly preventive. The primary objective of such business is cost reduction and financial profit. Nevertheless, it cannot be denied that loss of privacy and data is undesirable. We have to start analyzing costs, benefits, risks, probability, and impact (potential damage). That is not so evident in a technical branch as cybersecurity is.

An inseparable character of this play is malware. Let us motivate this thesis by listing several examples.

Firstly, one of the most prevalent malicious software is \emph{ransomware}. Its overall damage is estimated to be $\$20$ billion, increasing every year \cite{purplesec2021}. Though the social impact might be arguably even more significant than the financial side as there have been attacks targetting even healthcare organizations, which is further exacerbated by the fact that the first death following a ransomware attack was reported in 2020 \cite{Cimpanu2020}.

We can conclude that IoT malware is becoming more common, supported by Sonic Wall's 2019 report. That is caused by the insufficient protection of these small devices, for which we cannot provide complete malware protection. However, 127 new IoT devices are being connected to the internet every second, which leads us to an estimation that by the end of 2021, there will be 35 billion IoT devices connected to the internet \cite{TheIoTRu52:online}. This risk can not be mitigated easily, and malware elimination will play an even more significant role as it has so far.

Another convenient trend for malware is widespread encryption, which has become a standard in web traffic. Its main goal is the security of information. Considering this fact, the creators of malware have a lot to hide and secure from the protectors as well. The encryption might inform us that the source has something that nobody else should see. A long-lasting trend of such behaviour might be suspicious. We can check if there is a justified reason to encrypt the data, or we can at least make some conclusions about the source of the encrypted data. Nevertheless, that is not possible in the world where everything is encrypted.

In 2020, 94\% of malware was delivered by email \cite{Topcyber13:online}, and therefore the importance of phishing emails with malicious attachments and other social engineering techniques grows. It is cheaper to produce one sophisticated, convincing email to retrieve some information than an attempt to attack a highly protected network perimeter. It also might be used to distribute malware or other threats.

In 2020 AV-atlas \cite{AVATLASM39:online} recorded over 750 million malware samples, and moreover, at the end of April 2021, this number has increased to over 820 million malware samples. The majority of them are executable files attacking Windows devices.

Malware research continues, and it will undoubtedly do so until we can introduce a sufficiently universal and flexible solution that will be able to detect zero-day threats (unseen). We might find a solution among machine learning models, which are often involved. However, its challenge is interpretability and explainability, not only in cybersecurity. We face the problem that a model's performance is often significant, but we are not sure about the reason, and it is risky to deploy such a model to a situation where it can meet unseen data. High-quality security engineers do not have to be high-quality machine learning engineers. If we want to involve machine learning methods increasingly, we need to interpret and explain its predictions to combine cybersecurity knowledge with the results of the models and gain a better understanding. 


\section*{Goal}
The main objective of this thesis is to design a pipeline that has a malware file dataset as the input and a machine learning model and its explanation as the output. We want to go through the whole process, document each step, and report results. Our acquisition is the process itself, so it is described in detail for the reader to identify the problems we experienced and replicate or extend our setup.

From the assignment of this thesis, we can extract the following steps:
\begin{enumerate}
    \itemsep0em 
    \item Run several instances of \emph{CAPEv2} \cite{Cape} sandbox and solve their orchestration for this experiment
    \item Capture the behaviour of selected malware samples and store the results
    \item Learn the hierarchical multiple instance learning (\emph{HMill}) framework
    \item Analyze the captured data. Report the basic statistics and choose appropriate features and hidden states for further modelling
    \item Using HMill, create models and identify the artefacts corresponding to different malware behaviour, and report the results
    \item Investigate which parts of the \emph{CAPEv2} log are important to different malware behaviour
\end{enumerate}
\subsection*{In detail}
The first step implies using dynamic malware analysis to retrieve the input for our machine learning model. This intention originated after we downloaded a couple of thousands of sandbox \emph{JSON} reports from the internet and examined them. We observed that this use case might be challenging for our method and might demonstrate its capabilities.

The initial task is data collection. We are about to use MalwareBazaar\footnote{https://bazaar.abuse.ch/} as a public data source of malware samples. We chose it because of its free access with no claims for usage and a reasonable amount of samples. We aim at \emph{Portable Executable} (PE), which does not require any additional software running on the target machine. The sandbox we want to use is \emph{CAPEv2} \cite{Cape} because the first reports were also produced by this tool, and they are sufficient for analysis purposes.  It is a fork of a popular \emph{Cuckoo} sandbox which is no longer maintained. The sandbox has to be run in multiple instances to collect a sufficient number of samples.

The model we want to use is a \emph{hierarchical multiple instance learning} model. In \cite{Mandlik2020}, the authors showed that this model has good performance modelling \emph{JSON} documents. After further research, we decided to model the dependence of malware \emph{signatures} on behavioural features, both included in \emph{JSON} report. Signatures are the essential input for the original classification techniques used by the sandbox, and we want to see how well the model predicts them based on malware behaviour.

Finally, we will attempt to explain the predictions by choosing a minimal subset of features that contributes to the model's prediction the most. The explanation will be performed using the existing \emph{HMill} explainer. We can study the implementation of signatures and their true cause, which might help us with results evaluation. We expect that explanation of the model, the cause of which is in the report, should be a set containing this cause. As an example, we expect that if the original signature's cause is a specific API call, it should be presented in the explanation of the binary classifier for this signature. As authors of \emph{HMill} explainer mentioned, we hope that the explanation contains even something new. In other words, we expect that we could observe explanations that contain entries that are not the original cause. However, they might reasonably substitute it --- they are connected to the same effects. It is also possible to identify new signatures because all samples are malicious and the model might generalize based on different similarities in the training set.

\section*{Thesis structure}
The thesis is divided into two main parts. In the first part, we focus on the theoretical background of our method. In the second part, we present a specific setup, our results, and their discussion. More complex structures (images, tables) are part of the appendices. The attachments containing additional material are described in \ref{app:attach}.

The theoretical part starts with the malware analysis theory in chapter \ref{chap:analysis} where we break down the malware itself, the types of its analyses, and its output. We continue in the chapter \ref{chap:classification} which describes the machine learning formalism, cybersecurity context, and structured data (\emph{JSON} document) usage in machine learning. Finally, the chapters \ref{chap:hmill} and \ref{chap:expth} describe the particular methods used in our modelling and explaining experiments.

The second part consists of two chapters. Chapter \ref{chap:infrastructure} includes a description of the infrastructure and the data collection process. Chapter \ref{chap:results} presents the model and explainer setup, results, and their discussion.
	%\chapter{Úvod}
Výsledná struktura vaší práce a názvy a rozsahy jednotlivých kapitol se samozřejmě budou lišit podle typu práce a podle konkrétní povahy zpracovávaného tématu. 

\section{Jak používat tuto šablonu}

\chapter{Pokyny a návody k formátování textu práce}
Používat se dají všechny příkazy systému \LaTeX. Existuje velké množství volně přístupné dokumentace, tutoriálů, příruček a dalších materiálů v elektronické podobě. Výchozím bodem, kromě Googlu, může být stránka CSTUG (Czech Tech Users Group). Tam najdete odkazy na další materiály.  Vetšinou dostačující a přehledně organizovanou elektronikou dokumentaci najdete například na nebo.

Existují i různé nadstavby nad systémy \TeX{} a \LaTeX, které výrazně usnadní psaní textu zejména začátečníkům. Z mnoha možných uvádíme: Kile\footnote{\url{http://kile.sourceforge.net/}}, TexMaker\footnote{\url{http://www.xm1math.net/texmaker/}}, LyX\footnote{\url{http://www.lyx.org/}}.


\section{Tabulky}
Existuje více způsobů, jak sázet tabulky. Například je možno použít prostředí \verb|table|, které je velmi podobné prostředí \verb|figure|. 

\begin{table}
\begin{center}
\begin{tabular}{|c|l|l|}
\hline
\textbf{DTD} & \textbf{construction} & \textbf{elimination} \\
\hline
$\mid$ & \verb+in1|A|B a:sum A B+ & \verb+case([_:A]a)([_:B]a)ab:A+\\
&\verb+in1|A|B b:sum A B+ & \verb+case([_:A]b)([_:B]b)ba:B+\\
\hline
$+$&\verb+do_reg:A -> reg A+&\verb+undo_reg:reg A -> A+\\
\hline
$*,?$& the same like $\mid$ and $+$ & the same like $\mid$ and $+$\\
& with \verb+emtpy_el:empty+ & with \verb+emtpy_el:empty+\\
\hline
R(a,b) & \verb+make_R:A->B->R+ & \verb+a: R -> A+\\
 & & \verb+b: R -> B+\\
\hline
\end{tabular}
\end{center}
\caption{Ukázka tabulky}
\label{tab:tab1}
\end{table}

Zdrojový text tabulky \ref{tab:tab1} vypadá takto:
\begin{verbatim}
\begin{table}
\begin{center}
\begin{tabular}{|c|l|l|}
\hline
\textbf{DTD} & \textbf{construction} & \textbf{elimination} \\
\hline
$\mid$ & \verb+in1|A|B a:sum A B+ & \verb+case([_:A]a)([_:B]a)ab:A+\\
&\verb+in1|A|B b:sum A B+ & \verb+case([_:A]b)([_:B]b)ba:B+\\
\hline
$+$&\verb+do_reg:A -> reg A+&\verb+undo_reg:reg A -> A+\\
\hline
$*,?$& the same like $\mid$ and $+$ & the same like $\mid$ and $+$\\
& with \verb+emtpy_el:empty+ & with \verb+emtpy_el:empty+\\
\hline
R(a,b) & \verb+make_R:A->B->R+ & \verb+a: R -> A+\\
 & & \verb+b: R -> B+\\
\hline
\end{tabular}
\end{center}
\caption{Ukázka tabulky}
\label{tab:tab1}
\end{table}
\begin{table}
\end{verbatim}

A pokud máte svá data v CSV můžete použít některou z knihoven nabízených v   http://texblog.org/2012/05/30/generate-latex-tables-from-csv-files-excel/ 

\section{Odkazy v textu}
\subsection{Odkazy na literaturu}
Jsou realizovány příkazem \verb|\cite{odkaz}|. 

Seznam literatury je dobré zapsat do samostatného souboru a ten pak zpracovat programem bibtex (viz soubor \verb|reference.bib|). Zdrojový soubor pro \verb|bibtex| vypadá například takto:
\begin{verbatim}
@Article{Chen01,
  author  = "Yong-Sheng Chen and Yi-Ping Hung and Chiou-Shann Fuh",
  title   = "Fast Block Matching Algorithm Based on 
             the Winner-Update Strategy",
  journal = "IEEE Transactions On Image Processing",
  pages   = "1212--1222",
  volume  =  10,
  number  =   8,
  year    = 2001,
}

@Misc{latexdocweb,
  author  = "",
  title   = "{\LaTeX} --- online manuál",
  note    = "\verb|http://www.cstug.cz/latex/lm/frames.html|",
  year    = "",
}
...
\end{verbatim}

%11.12.2008, 3.5.2009
\textbf{Pozor:} Sazba názvů odkazů je dána Bib\TeX{} stylem\\ (\verb|\bibliographystyle{abbrv}|). 
%Budete-li používat české prostředí (\verb|\usepackage[czech]{babel}|), 
Bib\TeX{} tedy obvykle vysází velké pouze počáteční písmeno z názvu zdroje, 
ostatní písmena zůstanou malá bez ohledu na to, jak je napíšete. 
Přesněji řečeno, styl může zvolit pro každý typ publikace jiné konverze. 
Pro časopisecké články třeba výše uvedené, jiné pro monografie (u nich často bývá 
naopak velikost písmen zachována).

Pokud chcete Bib\TeX u napovědět, která písmena nechat bez konverzí 
(viz \texttt{title = "\{$\backslash$LaTeX\} -{}-{}- online manuál"} 
v~předchozím příkladu), je nutné příslušné písmeno (zde celé makro) uzavřít 
do složených závorek. Pro přehlednost je proto vhodné celé parametry 
uzavírat do uvozovek (\texttt{author = "\dots"}), nikoliv do složených závorek.

Odkazy na literaturu ve zdrojovém textu se pak zapisují:
% \begin{verbatim}
% % Podívejte se na %\cite{Chen01}, 
% % další detaily najdete na %\cite{latexdocweb}
% \end{verbatim}

Vazbu mezi soubory \verb|*.tex| a \verb|*.bib| zajistíte příkazem 
\verb|\bibliography{}| v souboru \verb|*.tex|.  V našem případě tedy zdrojový 
dokument \verb|thesis.tex| obsahuje příkaz\\
\verb|\bibliography{reference}|.

Zpracování zdrojového textu s odkazy se provede postupným voláním programů\\
\verb|pdflatex <soubor>| (případně \verb|latex <soubor>|), \verb|bibtex <soubor>| 
a opět\\ \verb|pdflatex <soubor>|.\footnote{První volání \texttt{pdflatex} 
vytvoří soubor s~koncovkou \texttt{*.aux}, který je vstupem pro program 
\texttt{bibtex}, pak je potřeba znovu zavolat program \texttt{pdflatex} 
(\texttt{latex}), který tentokrát zpracuje soubory s příponami \texttt{.aux} a 
\texttt{.tex}. 
Informaci o případných nevyřešených odkazech (cross-reference) vidíte přímo při 
zpracovávání zdrojového souboru příkazem \texttt{pdflatex}. Program \texttt{pdflatex} 
(\texttt{latex}) lze volat vícekrát, pokud stále vidíte nevyřešené závislosti.}


Níže uvedený příklad je převzat z dříve existujících pokynů studentům, kteří 
dělají svou diplomovou nebo bakalářskou práci v~Grafické skupině.\footnote{Několikrát 
jsem byl upozorněn, že web s těmito pokyny byl zrušen, proto jej zde přímo necituji. 
Nicméně příklad sám o sobě dokumentuje obecně přijímaný konsensus ohledně citací 
v~bakalářských a diplomových pracích na KP.} Zde se praví:
\begin{small}
\begin{verbatim}
...
j) Seznam literatury a dalších použitých pramenů, odkazy na WWW stránky, ...
 Pozor na to, že na veškeré uvedené prameny se musíte v textu práce 
 odkazovat -- [1]. 
Pramen, na který neodkazujete, vypadá, že jste ho vlastně nepotřebovali 
a je uveden jen do počtu. Příklad citace knihy [1], článku v časopise [2], 
stati ve sborníku [3] a html odkazu [4]: 
[1] J. Žára, B. Beneš;, and P. Felkel. 
     Moderní počítačová grafika. Computer Press s.r.o, Brno, 1 edition, 1998. 
     (in Czech). 
[2] P. Slavík. Grammars and Rewriting Systems as Models for Graphical User 
     Interfaces. Cognitive Systems, 4(4--3):381--399, 1997. 
[3] M. Haindl, Š. Kment, and P. Slavík. Virtual Information Systems. 
     In WSCG'2000 -- Short communication papers, pages 22--27, Pilsen, 2000. 
     University of West Bohemia. 
[4] Knihovna grafické skupiny katedry počítačů: 
     http://www.cgg.cvut.cz/Bib/library/ 
\end{verbatim}
\end{small}
% \ldots{} abychom výše citované odkazy skutečně našli v (automaticky generovaném) seznamu literatury tohoto textu, musíme je nyní alespoň jednou citovat: Kniha \cite{kniha}, článek v~časopisu \cite{clanek}, příspěvek na konferenci \cite{sbornik}, www odkaz \cite{www}.

% Ještě přidáme další ukázku citací online zdrojů podle české normy. Odkaz na wiki o frameworcich \cite{wiki:framework} a ORM \cite{wiki:orm}. Použití viz soubor \verb|reference.bib|. V seznamu literatury by nyní měly být živé odkazy na zdroje. V \verb|reference.bib| je zcela nový typ publikace. Detaily dohledal a dodal Petr Dlouhý v dubnu 2010. Podrobnosti najdete ve zdrojovém souboru tohoto textu v komentáři u příkazu \verb|\thebibliography|.

\subsection{Odkazy na obrázky, tabulky a kapitoly}
\begin{itemize}
\item Označení místa v textu, na které chcete později čtenáře práce odkázat, se provede příkazem \verb|\label{navesti}|. Lze použít v prostředích \verb|figure| a  \verb|table|, ale též za názvem kapitoly nebo podkapitoly.
\item Na návěští se odkážeme příkazem \verb|\ref{navesti}| nebo \verb|\pageref{navesti}|.
\end{itemize}

\section{Rovnice, centrovaná, číslovaná matematika}
Jednoduchý matematický výraz zapsaný přímo do textu se vysází pomocí prostředí \verb|math|, resp. zkrácený zápis pomocí uzavření textu rovnice mezi znaky \verb|$|.

Kód \verb|$ S = \pi * r^2 $| bude vysázen takto: $ S = \pi * r^2 $.

Pokud chcete nečíslované rovnice, ale umístěné centrovaně na samostatné řádky, pak lze použít prostředí \verb|displaymath|, resp. zkrácený zápis pomocí uzavření textu rovnice mezi znaky \verb|$$|. Zdrojový kód: 
\begin{verb}
|$$ S = \pi * r^2 $$|
\end{verb}
bude pak vysázen takto:
$$ S = \pi * r^2 $$

Chcete-li mít rovnice číslované, je třeba použít prostředí \verb|eqation|. Kód:
\begin{verbatim}
\begin{equation}
  S = \pi * r^2
\end{equation}

\begin{equation}
  V = \pi * r^3
\end{equation}
\end{verbatim}
je potom vysázen takto:
\begin{equation}
  S = \pi * r^2
\end{equation}

\begin{equation}
  V = \pi * r^3
\end{equation}

\section{Kódy a algoritmy}
\subsection{Zdrojové kódy}
Chceme-li vysázet například část zdrojového kódu programu, hodí se prostředí \verb|verbatim|, které je bez formátování: 
\begin{verbatim}
         (* nickname2 *)
Lego> Refine in1
             (do_reg (nickname1 h));
Refine by  in1 (do_reg (nickname1 h))
   ?4 : pcdata
   ?5 : pcdata
          (* surname2 *)
Lego> Refine surname1 h;
Refine by  surname1 h
   ?5 : pcdata
          (* email2 *)
Lego> Refine undo_reg (email1 h);
Refine by  undo_reg (email1 h)
*** QED ***
\end{verbatim}

nebo se dá použít \verb|listings|, což je package, který umožňuje i syntax higlighting podle jazyka:

\lstset{language=Python} %následuje python kód a použije se python syntax
\begin{lstlisting}
	print 'Hello, world!'
\end{lstlisting}

a umožňuje načíst i přiložené zdrojové soubory:

\lstinputlisting[language=C]{code/hello.c}

\subsection{Algoritmy}
Pokud chcete v práci popsat obecné algoritmy s využitím pseudokódu, můžete použít knihovny \verb|algorithmicx| a \verb|algpseudocode|:
\begin{algorithmic}
\If {$i\geq maxval$}
    \State $i\gets 0$
\Else
    \If {$i+k\leq maxval$}
        \State $i\gets i+k$
    \EndIf
\EndIf
\end{algorithmic}

\section{Zkratky}
% V tomto textu používám několik zkratek, třeba 2D\nomenclature{2D}{Two-Dimensional} nebo \nomExpl{KNB}{Killing nanobots}. V úvodní části dokumentu můžete nastavit/upravit příkaz pro zadání zkratky, pokud např. chcete mít význam zkratky pod čarou. Všechny zkratky se vytisknou podle abecedy (\nom{ABC}{Zkratka pro abecedu}) v příloze \ref{apx:zkratky}, aby toto fungovalo musíte \cite{Amerini2020} vybudovat index pomocí příkazu:
\begin{verbatim}
		makeindex soubor.nlo -s nomencl.ist -o soubor.nls
\end{verbatim}

\section{České uvozovky}
V souboru \verb|k336_thesis_macros.tex| je příkaz \verb|\uv{}| pro sázení českých uvozovek. \uv{Text uzavřený do českých uvozovek.}

\chapter{Závěr}

\begin{itemize}
\item Zhodnocení splnění cílů DP/BP a  vlastního přínosu práce (při formulaci je třeba vzít v potaz zadání práce).
\item Diskuse dalšího možného pokračování práce.
\end{itemize} 


%%%%%%%%%%%%%%%%%%%%%%%%%% 
% Seznam literatury je v samostatnem souboru reference.bib. Ten
% upravte dle vlastnich potreb, potom zpracujte (a do textu
% zapracujte) pomoci prikazu bibtex a nasledne pdflatex (nebo
% latex). Druhy z nich alespon 2x, aby se poresily odkazy.

% originally following specification for bibliography formating was used
%\bibliographystyle{abbrv}

% Here is an improvment by Petr Dlouhy (April 2010).
% It is mainly for supervisors who expect Czech fomrating rules for references
% Additional feature is live url addresses to sources from your pdf file
% It requires the file csplainnat.bst (included in this sample zipfile).

	%\include{original/obrazky.tex}


	%Biblio
	\bibliographystyle{alpha}
	%\bibliographystyle{csplainnat}
	%bibliographystyle{plain}
	% \bibliographystyle{psc}
	%\bibliographystyle{apalike}
	{
	\bibliography{bib/Master_thesis}
	}

	%%%%%%%%%%%%%%%%%%%%%%%%%% 
	% APPENDIX
	\appendix	

	\printnomenclature
	\label{apx:zkratky}

	\chapter{Obsah přiloženého CD}
	\textbf{\large Tato příloha je povinná pro každou práci. Každá práce musí totiž obsahovat přiložené CD. Viz dále.}

	% Může vypadat například takto. Váš seznam samozřejmě bude odpovídat typu vaší práce. (viz \cite{infodp}):

	% \begin{figure}[h]
	% \begin{center}
	% \includegraphics[width=14cm]{figures/seznamcd}
	% \caption{Seznam přiloženého CD --- příklad}
	% \label{fig:seznamcd}
	% \end{center}
	% \end{figure}

	% Na GNU/Linuxu si strukturu přiloženého CD můžete snadno vyrobit příkazem:\\ 
	% \verb|$ tree . >tree.txt|\\
	% Ve vzniklém souboru pak stačí pouze doplnit komentáře.

	% Z \textbf{README.TXT} (případne index.html apod.)  musí být rovněž zřejmé, jak programy instalovat, spouštět a jaké požadavky mají tyto programy na hardware.

	% Adresář \textbf{text}  musí obsahovat soubor s vlastním textem práce v PDF nebo PS formátu, který bude později použit pro prezentaci diplomové práce na WWW.

\end{document}
